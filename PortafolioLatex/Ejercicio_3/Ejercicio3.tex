\documentclass{article}
\usepackage{amsmath}
\usepackage{amssymb}
\usepackage{graphicx}

%Primero será necesario inicializar el documento%
%Inicio documento%
\begin{document}
%Ahora se comenzara a hacer la portada para comenzar con el trabajo%
%Inicio Portada%
    \begin{titlepage}
        %Centraremos el texto para comenzar%
        \centering
        %Fotografia del escudo de la escuela%
        {\includegraphics[width=0.2\textwidth]{logo.png} \par}
        %Texto de la portada%
        {\bfseries\LARGE\textit{Universidad Nacional Autónoma de México}\par}
        %Vspace sera el espacio que existe entre un parrafo y el siguiente%
        \vspace{1cm}

        {\scshape\Large Facultad de Estudios Superiores Acatlán \par}
        \vspace{3cm}

        {\scshape\Huge Ejercicio 3: Aún r}
        \vspace{3cm}

        {\slshape\Large Materia: Métodos Numéricos $1$ \par}
        \vfill

        {\Large Autor: Díaz Valdez Fidel Gilberto \par}
        {\Large Número de cuenta: 320324280 \par}

        \vfill
        {\Large Septiembre 2023 \par}

    \end{titlepage}

\section{Propósito}
Hacer uso del método numérico de Newton-Raphson para aplicarlo en la problemática a resolver para así encontrar una solución satisfactoria al problema además de ganar soltura y experiencia tanto en la utilización de este método, como en la aplicación de otras dependiendo de lo que se requiera y las herramientas que se dispongan. 

\section{Instrucciones}
Según la ecuación de Van der Waals para algún gas real encontrar el valor de $V$. 

El gas en cuesitón es dimetilamina y la ecuación la siguiente:
$$ (P+\frac{a}{V^2})(V-b)=RT $$

Los datos proporcionados:
\begin{itemize}
    \item $P$ = Presión en $atm$ (atmosférica) = $50 atm$
    \item $T$ = Temperatura en $K$ (Kelvin) = $75$ $C$º
    \item $R$ = $0.08205 atm-L/(gmolK)$
    \item $V$ = Volumen molar del gas en $L/mol$
    \item $a$ = Constante del gas dimetilamina = $37.49$
    \item $b$ = Constante del gas dimetilamina = $0.197$
\end{itemize}

\section{Fórmula}
El siguiente paso es sustituir los valores que nos proporcionan en la misma ecuación.

Pero antes de hacer esto es necesario pasar la temperatura a Kelvin ya que se nos fue proporcionada en Celsius.

La fórmula es esta: $75$Cº$+243.15K= 348.15K$


El resultado de sustitur los datos es:
$$ (P+\frac{a}{V^2})(V-b)=RT $$




\end{document}
%Fin Documento%
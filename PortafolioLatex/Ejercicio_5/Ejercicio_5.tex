\documentclass{article}
\usepackage{graphicx}
\usepackage{amsmath}
\usepackage{mathtools}
\usepackage{cancel}


\begin{document}
\begin{titlepage}
   \centering


   {\includegraphics[width=2.5cm]{logo.png}\par}


   {\texttt{\bfseries \LARGE Universidad Nacional Autónoma de México} \par}
   \vspace{1cm}


   {\itshape \Large \bfseries Facultad de Estudios Superiores Acatlán \par}
   \vspace{3cm}


   {\scshape \Huge Ejercicio 5: Inversa/determinante\par}
   \vspace {3cm}


   {\slshape \Large Materia: Métodos Numéricos \par}
   \vspace{2cm}


   {\Large Autor: Díaz Valdez Fidel Gilberto\par}
   {\Large Número de cuenta: 320324280\par}
   \vfill


   {\itshape Octubre 2023 \par}
\end{titlepage}


\section{Propósito}
La idea principal es desempolvar las nociones antes vistas en Álgebra Superior sobre las
operaciones matriciales, que nos ayudan a poder determinar si la matriz en cuestión tendrá
o no solución única o de algún otro tipo y a partir de eso hacer uso de los métodos antes
estudiados para la obtención de esa solución.


\section{Instrucciones}
Se nos proporciona la siguiente matriz:
\begin{equation*}
   A =
   \begin{bmatrix}
       1   &  4  & $-2$ &  0 \\
      $-3$   & $-2$  &  0 &  1 \\
       3   &  2  &  1 & $-1$ \\
       2   & $-2$ &  3 &  4 \\
   \end{bmatrix}
\end{equation*}


Obtener:
\begin{itemize}
   \item Matriz de menores
   \item Matriz de cofactores
   \item Determinante
   \item Matriz Adjunta
   \item Matriz Inversa
\end{itemize}


\section{Planteamiento}
Ya tenemos todo lo que debemos obtener, observando detenidamente podemos observar
cuál será la forma más óptima de comenzar para no tener que trabajar de más.


Primero para la obtención de la matriz de menores tenemos que saber lo que es. El
menor $M_{ij}$ es el determinante de la matriz resultante de eliminar el renglón $i$
y la columna $j$. En el siguiente ejemplo se tendrá una matriz de $3x3$ denotada por
$A_{3x3}$ y se verá de esta forma.
\begin{equation*}
   A =
   \begin{bmatrix}
       a_{1,1}   &  a_{1,2}  &a_{1,3} \\
       a_{2,1} & a_{2,2} &  a_{2,3} \\
       a_{3,1} &  a_{3,2}  & a_{3,3} \\
   \end{bmatrix}
\end{equation*}


Si digamos que se quiere obtener el menor de $A_{1,1}$ entonces se suprimirá el renglón
$1$ y la columna y el menor será el determinante de esta matriz resultante:


\begin{equation*}
   A =
   \begin{bmatrix}
       a_{1,1}  \makebox(-15,0){\rule[-15ex]{0.8pt}{10ex}}& a_{1,2} \makebox(-15,0){\rule[1ex]{19ex}{0.8pt}} & a_{1,3} \\
       a_{2,1} & a_{2,2} &  a_{2,3} \\
       a_{3,1} &  a_{3,2}  & a_{3,3} \\
   \end{bmatrix}
\end{equation*}


Entonces el menor de $A_{1,1}$ será el determinante de esta submatriz:
\begin{equation*}
   A_s =
   \begin{bmatrix}
       a_{2,2} &  a_{2,3} \\
       a_{3,2}  & a_{3,3}
   \end{bmatrix}
\end{equation*}


Este proceso se repetirá con todos los elementos de la matriz.


Para continuar debemos obtener la matriz de cofactores la cual se obtiene con la siguiente
fórmula:
$$A_{ij} = (-1)^{i+j}M_{ij}$$


Donde $M_{ij}$ denota el menor de la posición $i,j$ por lo que al tener la matriz
completa de menores, ya nos encontramos en la mejor condición para calcular la
matriz de cofactores.


Para el cálculo del determinante fácilmente podemos usar el método de los cofactores
ya que ya poseemos la matriz de los mismos. Esta funciona al multiplicar todos los
cofactores de un renglón o columna escogido, sumarlos y este dígito será nuestro
determinante.


Para la obtención de la matriz adjunta solo tendremos que encontrar la matriz de cofactores
y obtener la transpuesta de esta ya la definición de matriz adjunta es esta:
$$adj(A) = [cof(A)]^T$$


Por último tenemos que obtener la matriz inversa, existen varios métodos, pero al haber
hecho todos estos cálculos y teniendo tantos materiales no podíamos desperdiciarlos, es
por esto que se usará la siguiente fórmula para la obtención de esta matriz:
$$A^{-1} = \frac{adj(A)}{det(A)}$$


Como se puede ver solo necesitamos la adjunta y el determinante de $A$, las cuales ya
encontramos, por lo que ya no tenemos que hacer algo muy complejo.

\section{Ejecución}
\subsection{Matriz de menores}
Para la obtención de esta simplemente debemos seguir el mismo método antes descrito.
\begin{equation*}
   A =
   \begin{bmatrix}
       1   &  4  & $-2$ &  0 \\
      $-3$   & $-2$  &  0 &  1 \\
       3   &  2  &  1 & $-1$ \\
       2   & $-2$ &  3 &  4 \\
   \end{bmatrix}
\end{equation*}


Primero calcularemos el menor de $A_{1,1}$.
$$M_{1,1} = det \begin{bmatrix}
   $-2$ & 0 & 1 \\
   2 & 1 & $-1$ \\
   $-2$ & 3 & 4
\end{bmatrix}$$


Debemos calcular entonces la determinante de esa submatriz, pero como ya sabemos esta se
Se calcula al sumar entre sí todo un renglón o columna de la matriz de cofactores. Como solo
necesitamos el determinante, es decir, un renglón o columna de cofactores así procederemos
elegimos primero un renglón o una columna para calcular los cofactores, en este caso yo elijo
el primer renglón, por lo tanto es necesario encontrar el cofactor de $-2$ , $0$ y $1$.


\begin{equation*}
   cof_{1,1} = (-1)^{1+1}M_{1,1} =1 M_{1,1}
\end{equation*}


Como se puede ver es necesario volver a calcular un menor, pero ahora al ser el menor de
$(1,1)$ se hace más pequeña nuestra submatriz, es la siguiente:
\begin{equation*}
   A_{sub} =
   \begin{bmatrix}
       1 & $-1$ \\
       3 &  4
   \end{bmatrix}
\end{equation*}
Pero el cálculo del determinante de una matriz de $2x2$ es el siguiente:
\begin{equation*}
   A =
   \begin{bmatrix}
       a_{1,1} &a_{1,2} \\
       a_{2,1} & a_{2,2}
   \end{bmatrix}
   \rightarrow
   det(A) = (a_{1,1}*a_{2,2}) - (a_{2,1}*a_{1,2})
\end{equation*}


Por lo tanto el determinante de nuestras submatriz es el siguiente:
$$det(A_{sub}) = (1*4)-(-1*3) = 4+3 = 7$$
Entonces el $cof_{1,1} = 1 * 7 = 7$, que es lo mismo que decir que el cofactor de $-2 = 7$
se deberá proceder de la misma forma para la obtención del cofactor de $0$ y $-1$.


\subsubsection*{Cofactor de $0$}
$$cof_{1,2} = (-1)^{1+2} M_{1,2} = -1(M_{1,2})$$
$$M_{1,2} = det 
\begin{bmatrix}
   2 & $-1$ \\
   $-2$ &  4
\end{bmatrix} = (2*4)-(-1*-2) = 8-2 = 6 $$


Por lo tanto $cof_{1,2} = -1(6) = -6$


\subsubsection*{Cofactor de $-1$}
$$cof_{1,3} = (-1)^{1+3} M_{1,3} = 1(M_{1,3})$$
$$M_{1,3} = det
\begin{bmatrix}
   2 & 1\\
   $-2$ &  3
\end{bmatrix} = (2*3)-(-2*1) = 6+2  = 8$$


Por lo tanto $cof_{1,3} = 1(8) = 8$


\subsubsection*{Cofactor buscado}


Con los cofactores del renglón elegido de la matriz resultante para el cálculo del
menor $M_{1,1}$ de la matriz original elegida $A$ estamos en condiciones de calcularlo,
debemos recordar que para el cálculo del determinante a través de cofactores, no se usa
solo los cofactores encontrados sino también el coeficiente del cofactor, en este caso
sería algo así:
$$M_{1,1} = det \begin{bmatrix}
   $-2$ & 0 & 1 \\
   2 & 1 & $-1$ \\
   $-2$ & 3 & 4
\end{bmatrix} = (-2)7+ (0)(-6)+ (1)8 = -14 +8 = -6 $$
Aquí como se puede ver $-2$ que multiplica a $7$ es el coeficiente del cofactor de su
posición, en este caso ${1,1}$ y lo mismo sucede con el $0$ y el $-6$.


Por lo tanto el menor $M_{1,1} = -6$, este mismo proceso se debe repetir hasta encontrar
el menor de todas las posiciones y así armar la matriz de menores, pero antes de eso me
Me gustaría aclarar algunos puntos que pueden agilizar el procedimiento.


\subsubsection*{Cofactores}


Como vimos para calcular el cofactor de una posición es necesario solo el menor y
multiplicarlo por $(-1)^{i+j}$ donde ${i,j}$ representan la posición, pero si observamos
más detenidamente esto nos podemos dar cuenta que el signo, osea $(-1)^{i+j}$, cambia
dependiendo de si es par o no.


Un ejemplo de esto. Por el momento no nos preocuparemos por el menor, si buscamos el
cofactor de la posición ${1,1}$, sabemos que tenemos que aplicar $cof_{1,1}=(-1)^{1+1} M_{1,1}$
entonces el $M_{1,1}$ se verá afectado por la multiplicación de $(-1)^{1+1}$ pero al
tener exponente par este se convertirá en $1$ positivo, por lo que no afectará el valor de
$M_{1,1}$ y tampoco lo haría de manera muy grave si el exponente no fuese par, ya que
el resultado sería un $-1$, simplemente se cambiaría el signo del menor.


Realizando un análisis entonces podemos observar que el menor cambia de signo dependiendo
del valor de la suma de los subíndices ${i,j}$ pero de igual manera podemos ver que conforme
se avanza de posición la suma se alterna entre par e impar:
$${1,1} = 2 \rightarrow {1,2} = 3 \rightarrow {1,3} = 4$$


Cómo avanzan una unidad a la vez el valor de la suma de estos alterna de par e impar y lo mismo
sucede recorriendo las columnas ya que estas también tienen un avance de una unidad a la vez.
Haciendo este análisis podemos afirmar que la matriz de cofactores se ve algo así:
\begin{equation*}
   \begin{matrix}
       + & - & + & - \\
       - & + & - & + \\
       + & - & + & - \\
       - & + & - & +
   \end{matrix}
\end{equation*}


Este es un ejemplo de cómo se vería la matriz de cofactores de una matriz de 4x4 lo
único que haría falta es poner el menor de cada una de las posiciones en su lugar y
multiplicarlo por el respectivo signo de la matriz de cofactores encontrada.


Este análisis nos será útil cuando queramos obtener la matriz de cofactores.


\subsubsection*{Matriz de menores final}


Ya hemos mostrado cómo calcular el menor de una posición escogida, el proceso se deberá
repetir hasta encontrar el de todas las posiciones y cuando así se haga se tendrá esta
matriz:
\begin{equation*}
   M =
   \begin{pmatrix}
      $-6$ & $-14$ & $0$ & $-10$ \\
      $40$ & $35$ & $-50$ & $0$\\
      $-24$ & $-31$ & $50$ & $10$\\
      $-4$ & $-1$ & $0$ & $10$
   \end{pmatrix}
\end{equation*}
Ese fue el resultado de en cada posición de la matriz original encontrar su determinante
es decir:
$$M_{i,j} = det(A_{resultante})$$
Donde $A_{resultante}$ se refiere a la matriz resultante de suprimir el renglón $i$ y la
columna $j$.
Opte por no extenderme más y simplemente poner el ejemplo de cómo calcular un menor debido
a que esto es un repaso nada más, pero también por eso explique tanto lo que realizaba,
para que un solo ejemplo bastará.


\subsection{Matriz de Cofactores}
Esta matriz como ya explicamos en el \textsl{Planteamiento} y en la ejemplificación de
la obtención de la \texttt{matriz de menores}, es simplemente la misma matriz pero alternando
su signo debido a la naturaleza de su fórmula como ya se explicó al final de la subsección
\emph{Cofactores}.


Por lo tanto solo transcribimos la misma matriz de menores pero multiplicando por su
signo correspondiente el coeficiente que corresponda con la posición en la que se encuentra,
esto nos da de resultado la siguiente matriz:
\begin{equation*}
   cof(A) =
   \begin{pmatrix}
      $-6$ & $14$ & $0$ & $10$ \\
      $-40$ & $35$ & $50$ & $0$\\
      $-24$ & $31$ & $50$ & $-10$\\
      $4$ & $-1$ & $0$ & $10$
   \end{pmatrix}
\end{equation*}


\subsection{Determinante}
Gracias a que ya tenemos la matriz de cofactores es mucho más sencillo encontrar el determinante
de toda la matriz.


En el apartado de \emph{EL cofactor buscado} se muestra al final que a partir de los cofactores
encontrados de la matriz, solo será necesario elegir un renglón o columna y recorrerlo por
completo mientras multiplicas cada cofactor con el coeficiente de la posición que comparten, si
estoy en la posición ${1,1}$ multiplicó el $cof_{1,1}$ con $A_{1,1}$ siendo $A$ la matriz
de la que se derivó la de cofactores.


Con esto ya estamos en condiciones para el cálculo de la determinante ya que ya tenemos
la matriz de cofactores, solo hace falta elegir un renglón o columna para realizar las
operaciones, en lo personal yo elegiré el primer renglón porque es el que conlleva menos
operaciones.
$$det(A) = a_{1,1}\cdot cof_{1,1} + a_{1,2}\cdot cof_{1,2} + a_{1,3}\cdot cof_{1,3}+ a_{1,4}\cdot cof_{1,4}$$
$$det(A) = 1(-6) + 4(14) + (-2)0 + (0)-10 = -6 +56 + 0 + 0 = 50  \rightarrow det(A) = 50$$


\subsection{Matriz Adjunta}
De igual manera como vimos en el \emph{Planteamiento}, la adjunta será la misma matriz
que la matriz de cofactores pero intercambiando cada renglón por su columna, el primer
renglón será la primera columna, el segundo renglón su segunda columna, etc.


Matriz de Cofactores:
\begin{equation*}
   cof(A) =
   \begin{pmatrix}
      $-6$ & $14$ & $0$ & $10$ \\
      $-40$ & $35$ & $50$ & $0$\\
      $-24$ & $31$ & $50$ & $-10$\\
      $4$ & $-1$ & $0$ & $10$
   \end{pmatrix}
\end{equation*}


Matriz Adjunta:
\begin{equation*}
   adj(A) =
   \begin{pmatrix}
      $-6$ & $-40$ & $-24$ & $4$ \\
      $14$ & $35$ & $31$ & $-1$\\
      $0$ & $50$ & $50$ & $0$\\
      $10$ & $0$ & $-10$ & $10$
   \end{pmatrix}
\end{equation*}


\subsection{Matriz Inversa}
Gracias a todas las herramientas que hemos recolectado podemos calcular la inversa de la
matriz con la fórmula antes descrita, ya que poseemos todo lo necesario para su construcción.
$$A^{-1} = \frac{adj(A)}{det(A)}$$


$$A^{-1} = \frac{1}{50}=
\begin{pmatrix}
  $-6$ & $-40$ & $-24$ & $4$ \\
  $14$ & $35$ & $31$ & $-1$\\
  $0$ & $50$ & $50$ & $0$\\
  $10$ & $0$ & $-10$ & $10$
\end{pmatrix}$$


Multiplicando cada elemento de la matriz por $\frac{1}{50}$ tenemos finalmente la matriz
inversa.


$$A^{-1} =
\begin{pmatrix}
  -6 / 50 & \frac{-40}{50} & \frac{-24}{50} & \frac{4}{50}\\
   & & & \\
  \frac{14}{50} & \frac{35}{50} & \frac{31}{50} & \frac{-1}{50}\\
  & & & \\
  \frac{0}{50} & \frac{50}{50} & \frac{50}{50} & \frac{0}{50}\\
  & & & \\
  \frac{10}{50} & \frac{0}{50} & \frac{-10}{50} & \frac{10}{50}
\end{pmatrix}
\rightarrow A^{-1} =
\begin{pmatrix}
   \frac{-3}{25} & \frac{-4}{5} & \frac{-12}{25} & \frac{2}{25}\\
   & & & \\
  \frac{7}{25} & \frac{7}{10} & \frac{31}{50} & \frac{-1}{50}\\
  & & & \\
  0 & 1 & 1 & 0\\
  & & & \\
  \frac{1}{5} & 0 & \frac{-1}{5} & \frac{1}{5}
\end{pmatrix}$$
De esa manera ya logramos obtener la matriz inversa y todos los demás requerimientos antes
descritos en las instrucciones.


\section{Conclusión}
Fue una buena actividad para desempolvar las nociones sobre las matrices y cómo se compartan
por el uso que le daremos pronto en el curso. Me extendí más de lo que esperaba pero fue con
la intención de poder explicar más a fondo porque pasaban las cosas como lo hacían o se
comportan como lo hacen, para realmente tener una noción completa del comportamiento y
posibilidades en el entorno de las matrices.
\end{document}

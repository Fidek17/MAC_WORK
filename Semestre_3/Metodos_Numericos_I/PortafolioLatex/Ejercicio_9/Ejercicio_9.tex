\documentclass{article}
\usepackage{graphicx}
\usepackage{amsmath}
\usepackage{mathtools}
\usepackage[utf8]{inputenc}
\usepackage{changepage}
\providecommand{\abs}[1]{\lvert#1\rvert}
\providecommand{\norm}[1]{\lVert#1\rVert}
\usepackage{xcolor, colortbl}
\usepackage{array, multirow, multicol}
\usepackage{float}




\begin{document}
\begin{titlepage}
   \centering
   {\includegraphics[width=2.5cm]{logo.png}\par}
   {\texttt{\bfseries \LARGE Universidad Nacional Autónoma de México} \par}
   \vspace{1cm}
   {\itshape \Large \bfseries Facultad de Estudios Superiores Acatlán \par}
   \vspace{3cm}
   {\scshape \Huge Ejercicio 9: Métodos Iterativos \par}
   \vspace {3cm}
   {\slshape \Large Materia: Métodos Numéricos \par}
   \vspace{2cm}
   {\Large Autor: Díaz Valdez Fidel Gilberto\par}
   {\Large Número de cuenta: 320324280\par}
   \vfill
   {\itshape Noviembre 2023 \par}
\end{titlepage}


\section{Propósito}
Conocer la características de los métodos iterativos para su implementación en la práctica una vez
familiarizados con la ventajas y desventajas de cada uno de los métodos para la elección de uno al
momento de enfrentarse con un problema.


\section{Indicaciones}
Sea el siguiente sistema de ecuaciones:
\begin{equation*}
   \begin{matrix}
       2x_1 -x_2 + 10x_3 - x_4 = 45 \\
       -x_1 + 8x_2 - x_3 + 3x_4 = 15 \\
       2x_1 + 3x_2 - x_3 +8x_4 = -39 \\
       9x_1 -x_2 +2x_3 - 2x_4 = -11
   \end{matrix}
\end{equation*}
\begin{itemize}
   \item Resolver por los métodos de Jacobi, Gauss-Seidel y Relajación.
   \item Verificar que la matriz sea EDD, sino reordenar los renglones.
   \item Tomar como vector inicial el vector de unos.
   \item Obtener una solución con una tolerancia de $0.00003$ evaluada con la norma espectral. 
\end{itemize}


\section{Preparativos}
Para comenzar sabemos que tenemos nuestra forma matricial extendida de la siguiente forma:
\begin{equation*}
   \begin{bmatrix}
       2 & $-1$ & 10 & $-1$ & \vdots & 45\\
       $-1$ & 8 & $-1$ & 3 & \vdots & 15 \\
       2 & 3 & $-1$ & 8 & \vdots & -39 \\
       9 & $-1$ & 2 & $-2$ & \vdots & -11
   \end{bmatrix}
\end{equation*}
Como nos podemos dar cuenta, antes de hacer nada, es evidente que no se trata de una matriz \emph{EDD},
ya que no se cumple para todos los renglones que el elemento de la diagonal en valor absoluto es mayor
a la suma en valor absoluto de los elementos restantes del renglón, es decir:
\begin{equation*}
   \abs{a_{ii}} > \sum_{j_1}^{n} \abs{a_{ij}} j \neq i
\end{equation*}
Gracias a este tendremos que ordenar a la matriz de manera en que sea \emph{EDD}, es decir, que todos
los elementos de la diagonal sean mayores a la suma de sus elementos restantes en el renglón, a continuación
intercambiaremos renglones uno a la vez:
\begin{equation*}
   \begin{bmatrix}
       2 & $-1$ & 10 & $-1$ & \vdots & 45\\
       $-1$ & 8 & $-1$ & 3 & \vdots & 15 \\
       2 & 3 & $-1$ & 8 & \vdots & -39 \\
       9 & $-1$ & 2 & $-2$ & \vdots & -11
   \end{bmatrix}
   \rightarrow
   \begin{bmatrix}
       \colorbox{lime} 9 & $-1$ & 2 & $-2$ & \vdots & -11\\
       $-1$ & \colorbox{lime} 8 & $-1$ & 3 & \vdots & 15 \\
       2 & 3 & $-1$ & 8 & \vdots & -39 \\
       2 & $-1$ & 10 & $-1$ & \vdots & 45
   \end{bmatrix}
   \rightarrow
   \begin{bmatrix}
       \colorbox{lime} 9 & $-1$ & 2 & $-2$ & \vdots & -11\\
       $-1$ & \colorbox{lime} 8 & $-1$ & 3 & \vdots & 15 \\
       2 & $-1$ & \colorbox{lime}{10} & $-1$ & \vdots & 45\\
       2 & 3 & $-1$ & \colorbox{lime}8 & \vdots & -39
   \end{bmatrix}
\end{equation*}
Como se puede observar de esa manera la matriz ya es una matriz \emph{Estricta
Dominante Diagonalmente}.


Cabe recalcar o puntualizar que ahora la matriz tiene una forma de esta manera: $Ax = b$,
pero nosotros buscamos que tenga una forma de $x = Tx + C$ debido a que es la forma que nos
permite dar valores distintos en cada iteración, esto lo lograremos al despejar el vector $x$
de la siguiente forma:
\begin{equation*}
   \begin{matrix}
       x_1 = \frac{x_2}{9} - \frac{2x_3}{9} +\frac{2x_4}{9} - \frac{11}{9}\\
       \\
       x_2 = \frac{x_1}{8} + \frac{x_3}{8} - \frac{3x_4}{8} + \frac{15}{8}\\
       \\
       x_3 = -\frac{2x_1}{10} + \frac{x_2}{10} + \frac{x_4}{10} + \frac{45}{10}\\
       \\
       x_4 = - \frac{2x_1}{8} - \frac{3x_2}{8} + \frac{x_3}{8} - \frac{39}{8}
   \end{matrix}
\end{equation*}
Una vez se tiene de esta manera ya tenemos el vector $x$ pero si aún no es evidente se puede ver
en su forma matricial:
\begin{equation*}
   \begin{bmatrix}
       x^{,}_1 \\
       \\
       x^{,}_2 \\
       \\
       x^{,}_3 \\
       \\
       x^{,}_4
   \end{bmatrix}
   =
   \begin{bmatrix}
       0 & \frac{1}{9} & \frac{-2}{9} & \frac{2}{9} \\
       \\
       \frac{1}{8} & 0 & \frac{1}{8} & \frac{-3}{8} \\
       \\
       \frac{-2}{10} & \frac{1}{10} & 0 & \frac{1}{10} \\
       \\
       \frac{-2}{8} & \frac{-3}{8} & \frac{1}{8} & 0 \\
   \end{bmatrix}
   \begin{bmatrix}
       x_1 \\
       \\
       x_2 \\
       \\
       x_3 \\
       \\
       x_4
   \end{bmatrix}
   \begin{bmatrix}
       \frac{-11}{9} \\
       \\
       \frac{15}{8} \\
       \\
       \frac{45}{10} \\
       \\
       \frac{-39}{8}
   \end{bmatrix}
\end{equation*}
De esta manera ya se consiguió a la matriz de la forma $x = Tx + C$ donde la matriz $T$ es
mi matriz cuadrada, la matriz a la izquierda del igual será la matriz resultante de la operación
en donde la matriz que multiplica a la $T$ es nuestra matriz que cambia iteración a iteración.


Otra cosa que debemos tener en cuenta es la norma espectral y el cómo calcularla, esta se calcula así:
\begin{equation*}
   \norm{x-y}_\infty = max_{1\leq i \leq n} \abs{x-y}
\end{equation*}
Esto quiere que para calcular la norma espectral de un vector diferencia $x-y$ solo será el valor máximo
dentro del vector diferencia, es así como se calcula la norma espectral.




Una vez nuestra matriz se encuentra de esta manera ya estamos en condiciones de resolverla con los
distintos métodos.


\section{Ejecución}
\subsection{Método de Jacobi}
\subsubsection{Planteamiento}
En el método de Jacobi es necesario tener la matriz de la forma $x = Tx + C$, gracias a los preparativos
ya la poseemos y sólo hará falta implementar el método. Antes de entrar al cómo se hizo en la práctica quiero
dar una breve explicación de cómo funciona el método.


Vamos a comenzar con un vector $x$ de prueba, en este caso se pide que sea un vector de unos, se operará con este
vector de prueba y se guardará, en la siguiente iteración se usará el resultado anterior como el vector $x$ de prueba
y este proceso continuará hasta que la tolerancia al error que será medida con la norma espectral sea incumplida.


Se dice que este proceso es de ecuaciones simultáneas, ya que no se necesita de resultados anteriores para continuar
sino más bien se puede continuar haciendo todas las ecuaciones al mismo tiempo.


\subsubsection{Procedimiento}
Tomemos como ejemplo la primera iteración en donde el vector de prueba $x$ es de unos:
\begin{equation*}
   \begin{bmatrix}
       $-1.111111$ \\
       \\
       1.75000 \\
       \\
       4.50000 \\
       \\
       -5.37500
   \end{bmatrix}
   =
   \begin{bmatrix}
       0 & \frac{1}{9} & \frac{-2}{9} & \frac{2}{9} \\
       \\
       \frac{1}{8} & 0 & \frac{1}{8} & \frac{-3}{8} \\
       \\
       \frac{-2}{10} & \frac{1}{10} & 0 & \frac{1}{10} \\
       \\
       \frac{-2}{8} & \frac{-3}{8} & \frac{1}{8} & 0 \\
   \end{bmatrix}
   \begin{bmatrix}
       1 \\
       \\
       1 \\
       \\
       1 \\
       \\
       1
   \end{bmatrix}
   \begin{bmatrix}
       \frac{-11}{9} \\
       \\
       \frac{15}{8} \\
       \\
       \frac{45}{10} \\
       \\
       \frac{-39}{8}
   \end{bmatrix}
\end{equation*}
Ese es el resultado y la siguiente iteración tendrá como vector $x$ de prueba el vector $x$ resultado
de esta iteración. Las siguientes iteraciones se harán en la hoja de cálculo para agilizar el procedimiento.
La hoja tabla de operaciones es la siguiente:
\begin{figure}[h]
   \centering
   \begin{adjustwidth}{-2cm}{0cm}
       \begin{tabular}{| c | c | c | c | c | c | c | c | c |}
           \hline
           \multicolumn{9}{|c|}{Iteraciones} \\
           \hline
           k   &   0   &   1   &   2   &   3   &   4   &   5   &   6   &   7   \\
           \hline
           $x_1$   &   1   &   -1.111111   &   -3.222222   &   -2.754128   &   -3.080208   &   -2.962163   &   -3.016368   &   -2.993327   \\
           $x_2$   &   1   &   1.75   &   4.31424   &   3.7763   &   4.09745   &   3.96584   &   4.01861   &   3.99361   \\
           $x_3$   &   1   &   4.5   &   4.35972   &   5.10677   &   4.91422   &   5.02936   &   4.9863   &   5.00584   \\
           $x_4$   &   1   &   -5.375   &   -4.69097   &   -5.14232   &   -4.96423   &   -5.02721   &   -4.99298   &   -5.0046   \\
           \hline
           &   &   &   &   &   &   &   &   \\  % Espacio adicional
           \hline
           &   Espectral   &   6.375   &   2.56424   &   0.74705   &   0.32608   &   0.13161   &   0.0542   &   0.025   \\
           \hline
           &   Tolerancia   &   Continuar   &   Continuar   &   Continuar   &   Continuar   &   Continuar   &   Continuar   &   Continuar   \\
           \hline
       \end{tabular}
   \end{adjustwidth}
\end{figure}


\begin{figure}[h]
   \centering
   \begin{adjustwidth}{-2cm}{0cm}
       \begin{tabular}{| c | c | c | c | c | c | c | c |}
           \hline
           \multicolumn{8}{|c|}{Iteraciones} \\
           \hline
           8   &   9   &   10  &   11  &   12  &   13  &   14  &   15   \\ \hline
           -3.003029   &   -2.99877    &   -3.00055    &   -2.999772   &   -3.0001 &   -2.999958   &   -3.000018   &   -2.999992   \\
           4.00329 &   3.99877 &   4.00058 &   3.99977 &   4.0001  &   3.99996 &   4.00002 &   3.99999 \\
           4.99757 &   5.00108 &   4.99955 &   5.0002  &   4.99992 &   5.00004 &   4.99998 &   5.00001 \\
           -4.99854    &   -5.00078    &   -4.99971    &   -5.00014    &   -4.99995    &   -5.00002    &   -4.99999    &   -5  \\
               &       &       &       &       &       &       &       \\ \hline
           0.0097  &   0.00452 &   0.00181 &   0.00081 &   0.00034 &   0.00015 &   0.00006 &   0.00003 \\ \hline
           Continuar   &   Continuar   &   Continuar   &   Continuar   &   Continuar   &   Continuar   &   Continuar   &   Parar   \\ \hline
       \end{tabular}
   \end{adjustwidth}
\end{figure}


Como se puede observar se detienen las iteraciones porque se llegó a incumplir la tolerancia de la norma espectral tras 15 iteraciones, los datos finales
son redondeados y son el siguiente vector solución:
\begin{equation*}
   \begin{bmatrix}
       x_1 = -3 \\
       x_2 = 4 \\
       x_3 = 5 \\
       x_4 = -5
   \end{bmatrix}
\end{equation*}


\subsection{Método de Gauss-Seidel}
\subsubsection{Planteamiento}
Este método al ser de igual manera iterativo como lo fue el de Jacobi nos pide como requisito
base el tener el sistema de ecuaciones de la forma $x = Tx +C$, ya que esto ya lo tenemos podríamos
entrar de lleno al uso del método pero se explicara un poco de sus sútil diferencia con el método de
Jacobi.


En este método se busca mejorar, es decir, con cada iteración que nosotros hacemos nuestros valores
encontrados son cada vez mejores y más cercanos con el resultado buscado, la distinción del método de
Gauss-Seidel parte de esta razón medular.


Al momento de calcular una variable usaremos esta para la obtención de la siguiente variable en la siguiente
ecuación, en la posterior ecuación usaremos los dos mejores valores ya encontrados, es decir los últimos calculados
y este proceso continuará hasta haber calculado todas las ecuaciones. Es por esto que este método es más rápido,
se ahorra evaluaciones al utilizar el mejor valor en el momento en lugar de esperar al término de la iteración
para utilizarlo en la siguiente iteración.


Su formulación matricial es la siguiente:
\begin{equation*}
   x^{k} = T_gx^{(k-1)} + C_g
\end{equation*}
En donde:
\begin{itemize}
   \item $T_g = (D-L)^{-1}U$
   \item $C_g = (D-L)^{-1}b$
\end{itemize}
Puede a llegar a ser más evidente a su vez revisar cómo funciona ecuación por ecuación que sería de la siguiente
forma:
\begin{equation*}
   \textcolor{red}{x_i^{(k)}} = - \sum_{j=1}^{j=i-1}\frac{a_{ij}}{a_{ii}} \textcolor{red}{x_j^{(k)}} - \sum_{j = i+1}^{j = n}\frac{a_{ij}}{a_{ii}}\textcolor{cyan}{x_j^{(k-1)}} - \textcolor{red}{\frac{b_i}{a_{ii}}}
\end{equation*}
Esta forma al utilizarla nos dará como resultado la serie de ecuaciones que conseguimos en la sección de \emph{Preparativos} pero con la
diferencia de cual $x_i$ se utiliza, para se expondrá cómo queda la serie de ecuaciones y será tarea del lector comprobar que sea cierta.
\begin{equation*}
   \begin{matrix}
       \textcolor{red}{x_1^{(k)}} = \frac{1}{9}\textcolor{blue}{x_2^{(k-1)}} - \frac{2}{9}\textcolor{blue}{x_3^{(k-1)}} +\frac{2}{9} \textcolor{blue}{x_4^{(k-1)}}- \frac{11}{9}\\
       \\
       \textcolor{red}{x_2^{(k)}}  = \frac{1}{8}\textcolor{red}{x_1^{(k)}}  + \frac{1}{8}\textcolor{blue}{x_3^{(k-1)}} - \frac{3}{8}\textcolor{blue}{x_4^{(k-1)}} + \frac{15}{8}\\
       \\
       \textcolor{red}{x_3^{(k)}}  = -\frac{2}{10}\textcolor{red}{x_1^{(k)}} + \frac{1}{10}\textcolor{red}{x_2^{(k)}} + \frac{1}{10}\textcolor{blue}{x_4^{(k-1)}} + \frac{45}{10}\\
       \\
       \textcolor{red}{x_4^{(k)}}  = - \frac{2}{8}\textcolor{red}{x_1^{(k)}} - \frac{3}{8}\textcolor{red}{x_2^{(k)}} + \frac{1}{8}\textcolor{red}{x_3^{(k)}}  - \frac{39}{8}
   \end{matrix}
\end{equation*}


De esa manera más gráfica se puede observar cómo a partir del resultado de una ecuación anterior se encontrara el de una ecuación siguiente, es decir
se usa el mejor valor calculado desde el principio sin necesidad de esperar a la siguiente iteración.
\subsubsection{Procedimiento}
Una vez explicada la raíz del método ya estamos en condiciones de usar una hoja de cálculo como en el método de Jacobi pero configurandola para que use el valor
inmediato calculado anteriormente en la misma iteración, una vez hecho esto, su error se medirá con la norma espectral y la tolerancia será la misma, por
lo que la tabla resultante es la siguiente:
\begin{figure}[h]
   \centering
   \begin{adjustwidth}{-4cm}{0cm}
       \resizebox{20cm}{!}{
           \begin{tabular}{|c|c|c|c|c|c|c|c|c|c|c|c|}
               \hline
               \multicolumn{12}{|c|}{Iteraciones} \\
               \hline
               $k$ &   0   &   1   &   2   &   3   &   4   &   5   &   6   &   7   &   8   &   9   &   10  \\ \hline
               $x_1$   &   1   &   -1.111111111    &   -3.169097222    &   -3.007924564    &   -3.002629403    &   -3.000361893    &   -3.000062618    &   -3.000009945    &   -3.000001624    &   -3.000000263    &   -3.000000043    \\
               $x_2$   &   1   &   1.486111111 &   3.80015191  &   3.959780948 &   3.994057063 &   3.999010299 &   3.99984126  &   3.999974186 &   3.999995821 &   3.999999322 &   3.99999989  \\
               $x_3$   &   1   &   4.970833333 &   5.060518663 &   5.010041225 &   5.001763431 &   5.000284047 &   5.000046361 &   5.000007505 &   5.000001217 &   5.000000197 &   5.000000032 \\
               $x_4$   &   1   &   -4.533159722    &   -4.875217828    &   -4.981681561    &   -4.996893619    &   -4.999502883    &   -4.999919023    &   -4.999986895    &   -4.999997875    &   -4.999999656    &   -4.999999944    \\
               \hline
                   &       &       &       &       &       &       &       &       &       &       &       \\
               \hline
                   &   Espectral   &   5.533159722 &   2.31404 &   0.16117 &   0.03428 &   0.00495 &   0.00083 &   0.00013 &   0.00002 &   0   &   0   \\
                   &   Tolerancia  &   Continuar   &   Continuar   &   Continuar   &   Continuar   &   Continuar   &   Continuar   &   Continuar   &   Parar   &   Parar   &   Parar   \\
               \hline
           \end{tabular}
       }


   \end{adjustwidth}
  
\end{figure}


Como se puede ver se detiene en la iteración $9$ por que se incumple la tolerancia con respecto a la norma espectral. El vector solución restante redondeando es el mismo que en el método de Jacobi.
\begin{equation*}
   \begin{bmatrix}
       x_1 = -3 \\
       x_2 = 4 \\
       x_3 = 5 \\
       x_4 = -5
   \end{bmatrix}
\end{equation*}


\subsection{Método de Relajación}
\subsubsection{Planteamiento}
En este método se usarán algunos sistemas de ecuaciones ya encontrados anteriormente durante los \emph{Preparativos}
pero es importante mencionar la razón que hace tan distinto este método.
Para empezar tendremos que igualar cada ecuación a cero pero esto no será difícil ya que ya tenemos un sistema que nos ayudará.
\begin{equation*}
   \begin{matrix}
       x_1 = \frac{x_2}{9} - \frac{2x_3}{9} +\frac{2x_4}{9} - \frac{11}{9}\\
       \\
       x_2 = \frac{x_1}{8} + \frac{x_3}{8} - \frac{3x_4}{8} + \frac{15}{8}\\
       \\
       x_3 = -\frac{2x_1}{10} + \frac{x_2}{10} + \frac{x_4}{10} + \frac{45}{10}\\
       \\
       x_4 = - \frac{2x_1}{8} - \frac{3x_2}{8} + \frac{x_3}{8} - \frac{39}{8}
   \end{matrix}
\end{equation*}


Solo será necesario pasar las variables $x_i$ del otro lado pero necesitamos igualarlo con su residual \textcolor{red}{$R_i$},
este residual sabemos que tiene que ser cero para que se cumpla la igualdad, por lo tanto nos daremos a la tarea de que eso suceda,
comenzaremos el con un vector prueba que en este caso será de unos.


Consecuente a esto buscaremos el residual más alto en valor absoluto y buscaremos hacerlo cero, entonces se procede evaluando
las ecuaciones con la variable $x_i$ que pertenece al residual más alto con el valor de signo contrario de este residual y todas
las demás $x_i$ serán ceros, se evalúan todas las ecuaciones y una vez se tienen los resultados se restan con los anteriores,
de esta manera el residual escogido antes debe ser cero pero los residuales de las demás ecuaciones cambiarán, por lo que se continúa
repitiendo el proceso desde la elección del residual más grande en valor absoluto y se detendrá con la tolerancia en el error.


La tolerancia se calculará o medirá observando el residual más alto en valor absoluto después de operar, si ese residual es menor
a la tolerancia, significa que todos los demás residuales también lo son y por lo tanto se debe parar el método.


Antes de comenzar, nuestro sistema de ecuaciones se verá de la siguiente manera después de ser igualado con sus residuales \textcolor{red}{$R_i$}.


\begin{equation*}
   \begin{matrix}
       \textcolor{red}{R_1} = -x_1 + \frac{x_2}{9} - \frac{2x_3}{9} +\frac{2x_4}{9} - \frac{11}{9}\\
       \\
       \textcolor{red}{R_2} = \frac{x_1}{8} - x_2 + \frac{x_3}{8} - \frac{3x_4}{8} + \frac{15}{8}\\
       \\
       \textcolor{red}{R_3} = -\frac{2x_1}{10} + \frac{x_2}{10} - x_3+ \frac{x_4}{10} + \frac{45}{10}\\
       \\
       \textcolor{red}{R_4} = - \frac{2x_1}{8} - \frac{3x_2}{8} + \frac{x_3}{8} - x_4 - \frac{39}{8}
   \end{matrix}
\end{equation*}
Como ya tenemos conciencia sobre cómo funciona a nivel teórico el método y ya se explicó detenidamente, se continuará con la
muestra de la tabla que nos enseña cuántas iteraciones harían falta y cómo quedaría.
\newpage
\pagestyle{empty}
\begin{figure}[H]
   \centering
   \begin{adjustwidth}{-4cm}{0cm}
       \resizebox{20cm}{!}{
           \begin{tabular}{|c|c|c|c|c|c|c|c|c|c|c|}
               \hline
               \multicolumn{11}{|c|}{Iteraciones} \\ \hline
               &   $x_1$   &   $x_2$   &   $x_3$   &   $x_4$   &   $R_1$   &   $R_2$   &   $R_3$   &   $R_4$   &       &   ERROR   \\ \hline
               &   1   &   1   &   1   &   1   &   -2.11111    &   0.75    &   3.5 &   -6.375  &       &   Continuar   \\
               &       &       &       &   -6.375  &   -1.41667    &   2.39063 &   -0.6375 &   6.375   &       &   Continuar   \\ \rowcolor{lightgray}
               &       &       &       &       &   -3.52778    &   3.14063 &   2.8625  &   0   &   Suma    &       \\ \hline
               &   -3.52778    &       &       &       &   3.52778 &   -0.44097    &   0.70556 &   0.88194 &       &   Continuar   \\ \rowcolor{lightgray}
               &       &       &       &       &   0   &   2.69965 &   3.56806 &   0.88194 &   Suma    &       \\ \hline
               &       &       &   3.56806 &       &   -0.7929 &   0.44601 &   -3.56806    &   0.44601 &       &   Continuar   \\ \rowcolor{lightgray}
               &       &       &       &       &   -0.7929 &   3.14566 &   0   &   1.32795 &   Suma    &       \\ \hline
               &       &   3.14566 &       &       &   0.34952 &   -3.14566    &   0.31457 &   -1.17962    &       &   Continuar   \\ \rowcolor{lightgray}
               &       &       &       &       &   -0.44338    &   0   &   0.31457 &   0.14833 &   Suma    &       \\ \hline
               &   -0.44338    &       &       &       &   0.44338 &   -0.05542    &   0.08868 &   0.11085 &       &   Continuar   \\ \rowcolor{lightgray}
               &       &       &       &       &   0   &   -0.05542    &   0.40324 &   0.25917 &   Suma    &       \\ \hline
               &       &       &   0.40324 &       &   -0.08961    &   0.05041 &   -0.40324    &   0.05041 &       &   Continuar   \\ \rowcolor{lightgray}
               &       &       &       &       &   -0.08961    &   -0.00502    &   0   &   0.30958 &   Suma    &       \\ \hline
               &       &       &       &   0.30958 &   0.0688  &   -0.11609    &   0.03096 &   -0.30958    &       &   Continuar   \\ \rowcolor{lightgray}
               &       &       &       &       &   -0.02081    &   -0.12111    &   0.03096 &   0   &   Suma    &       \\ \hline
               &       &   -0.12111    &       &       &   -0.01346    &   0.12111 &   -0.01211    &   0.04542 &       &   Continuar   \\ \rowcolor{lightgray}
               &       &       &       &       &   -0.03427    &   0   &   0.01885 &   0.04542 &   Suma    &       \\ \hline
               &       &       &       &   0.04542 &   0.01009 &   -0.01703    &   0.00454 &   -0.04542    &       &   Continuar   \\ \rowcolor{lightgray}
               &       &       &       &       &   -0.02418    &   -0.01703    &   0.02339 &   0   &   Suma    &       \\ \hline
               &   -0.02418    &       &       &       &   0.02418 &   -0.00302    &   0.00484 &   0.00604 &       &   Continuar   \\ \rowcolor{lightgray}
               &       &       &       &       &   0   &   -0.02005    &   0.02822 &   0.00604 &   Suma    &       \\ \hline
               &       &       &   0.02822 &       &   -0.00627    &   0.00353 &   -0.02822    &   0.00353 &       &   Continuar   \\ \rowcolor{lightgray}
               &       &       &       &       &   -0.00627    &   -0.01653    &   0   &   0.00957 &   Suma    &       \\ \hline
               &       &   -0.01653    &       &       &   -0.00184    &   0.01653 &   -0.00165    &   0.0062  &       &   Continuar   \\ \rowcolor{lightgray}
               &       &       &       &       &   -0.00811    &   0   &   -0.00165    &   0.01577 &   Suma    &       \\ \hline
               &       &       &       &   0.01577 &   0.0035  &   -0.00591    &   0.00158 &   -0.01577    &       &   Continuar   \\ \rowcolor{lightgray}
               &       &       &       &       &   -0.0046 &   -0.00591    &   -0.00008    &   0   &   Suma    &       \\ \hline
               &   -0.0046 &       &       &       &   0.0046  &   -0.00058    &   0.00092 &   0.00115 &       &   Continuar   \\ \rowcolor{lightgray}
               &       &       &       &       &   0   &   -0.00649    &   0.00085 &   0.00115 &   Suma    &       \\ \hline
               &       &   -0.00649    &       &       &   -0.00072    &   0.00649 &   -0.00065    &   0.00243 &       &   Continuar   \\ \rowcolor{lightgray}
               &       &       &       &       &   -0.00072    &   0   &   0.0002  &   0.00358 &   Suma    &       \\ \hline
               &       &       &       &   0.00358 &   0.0008  &   -0.00134    &   0.00036 &   -0.00358    &       &   Continuar   \\ \rowcolor{lightgray}
               &       &       &       &       &   0.00008 &   -0.00134    &   0.00055 &   0   &   Suma    &       \\ \hline
               &       &   -0.00134    &       &       &   -0.00015    &   0.00134 &   -0.00013    &   0.0005  &       &   Continuar   \\ \rowcolor{lightgray}
               &       &       &       &       &   -0.00007    &   0   &   0.00042 &   0.0005  &   Suma    &       \\ \hline
               &       &       &       &   0.0005  &   0.00011 &   -0.00019    &   0.00005 &   -0.0005 &       &   Continuar   \\ \rowcolor{lightgray}
               &       &       &       &       &   0.00004 &   -0.00019    &   0.00047 &   0   &   Suma    &       \\ \hline
               &       &       &   0.00047 &       &   -0.0001 &   0.00006 &   -0.00047    &   0.00006 &       &   Continuar   \\ \rowcolor{lightgray}
               &       &       &       &       &   -0.00007    &   -0.00013    &   0   &   0.00006 &   Suma    &       \\ \hline
               &       &   -0.00013    &       &       &   -0.00001    &   0.00013 &   -0.00001    &   0.00005 &       &   Continuar   \\ \rowcolor{lightgray}
               &       &       &       &       &   -0.00008    &   0   &   -0.00001    &   0.00011 &   Suma    &       \\ \hline
               &       &       &       &   0.00011 &   0.00002 &   -0.00004    &   0.00001 &   -0.00011    &       &   Continuar   \\ \rowcolor{lightgray}
               &       &       &       &       &   -0.00006    &   -0.00004    &   0   &   0   &   Suma    &       \\ \hline
               &   -0.00006    &       &       &       &   0.00006 &   -0.00001    &   0.00001 &   0.00001 &       &   Continuar   \\ \rowcolor{lightgray}
               &       &       &       &       &   0   &   -0.00005    &   0.00001 &   0.00001 &   Suma    &       \\ \hline
               &       &   -0.00005    &       &       &   -0.00001    &   0.00005 &   0   &   0.00002 &       &   Continuar   \\ \rowcolor{lightgray}
               &       &       &       &       &   -0.00001    &   0   &   0   &   0.00003 &   Suma    &       \\ \hline
               &       &       &       &   0.00003 &   0.00001 &   -0.00001    &   0   &   -0.00003    &       &   Continuar   \\ \rowcolor{lightgray}
               &       &       &       &       &   0   &   -0.00001    &   0.00001 &   0   &   Suma    &       \\ \hline
               &       &   -0.00001    &       &       &   0   &   0.00001 &   0   &   0   &       &   Parar   \\ \rowcolor{lightgray}
           Resultado:  &   -3.000000142    &   4.000001311 &   4.999993184 &   -5.000005815    &       &       &       &       &       &       \\ \hline
           \end{tabular}
       }
   \end{adjustwidth}
  
\end{figure}
\newpage
\subsubsection{Procedimiento}
Como se puede observar, la tabla de iteraciones es mucho más larga que en los métodos anteriores, es importante mencionar que esto es incluso así
aún cuando su criterio de paro, hablamos de la tolerancia al error. También es importante puntualizar que al final si se redondea el
vector solución encontrado es el mismo al de los otros métodos, significando esto que se logró el cometido.
\begin{equation*}
   \begin{bmatrix}
       x_1 = -3 \\
       x_2 = 4 \\
       x_3 = 5 \\
       x_4 = -5
   \end{bmatrix}
\end{equation*}


\section{Conclusión}
Al momento de realizar esta tarea yo considero que los métodos más útiles para considerarse de naturaleza iterativa, que
trae consigo la facilidad de programar el método o implementarlo en una hoja de cálculo se encuentran presentes sólo en el
método de \emph{Jacobi} y en el de \emph{Gauss-Seidel}. Cabe puntualizar la estrecha relación que estos dos comparten, su
planteamiento es casi el mismo si solo exceptuamos como al final uno se puede trabajar simultáneamente mientras otro no.


Considero que por esa simple razón son los mejores métodos de los que ocupamos, fue sencillo implementarlos y no es tan complicado
programarlos, el cual se utilice ya dependerá en gran parte si se necesita hacer las operaciones de manera simultánea o si es
es posible hacerlo desde una sola práctica singular. Pero por otro lado tenemos al método de \emph{Relajación}, desde mi punto de
vista considero que es el método con menor naturaleza iterativa debido a su difícil implementación, es mucho más sencillo realizar el
método a mano que tener que configurar una hoja de cálculo de manera muy específica o un programa muy complejo, sin contar que toma de
muchas más iteraciones y se pudo ver reflejado en este caso.


Aquí es cuando considero que no es tan útil ya que, ¿De qué nos sirve un método iterativo si es más sencillo hacerlo a mano?, la
ventaja de este tipo de métodos es la implementación de herramientas computacionales y este método no está tan predispuesto a estas.


En conclusión diría que los métodos más eficaces, hablando de naturaleza iterativa, son \emph{Gauss-Seidel y Jacobi}, mientras
que el método de \emph{Relajación}, al momento de realizar este trabajo, no parece el más útil.
\end{document}





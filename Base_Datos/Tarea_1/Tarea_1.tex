\documentclass{article}
\usepackage{graphicx}

\begin{document}
\begin{titlepage}
    \centering
    {\includegraphics[width=2.5cm]{logo.png}\par}
    {\texttt{\bfseries \LARGE Universidad Nacional Autónoma de México} \par}
    \vspace{1cm}
    {\itshape \Large \bfseries Facultad de Estudios Superiores Acatlán \par}
    \vspace{3cm}
    {\scshape \Huge Tarea 1: Introducción a las bases de Datos \par}
    \vspace {3cm}
    {\slshape \Large Materia: Base de Datos \par}
    \vspace{2cm}
    {\Large Autor: Díaz Valdez Fidel Gilberto\par}
    {\Large Número de cuenta: 320324280\par}
    \vfill
    {\itshape Febrero 2024 \par}
\end{titlepage}

\section*{Instrucciones:}
1. Responde las siguientes preguntas, usa lo que vimos  en clase y expresa las respuestas con tus propias palabras.

\begin{itemize}
    \item ¿Qué es un dato?
    \item ¿Qué es una Base de datos?
    \item Menciona una ventaja de las BD
    \item Menciona  una desventaja de las BD
\end{itemize}

\subsection{¿Qué es un dato?}
Se trata de una estructura elemental de información tan pequeña que es dificil de dividir aún más ya que esto 
traeria consigo una perdida parcial de la información total. También cabe destacar que es información que al ser 
tan elemental y tan atomica realmente parece no tener un significado real o útil hasta el momento que se junta 
con otro conjunto de datos.


\subsection{¿Qué es una Base de datos?}
Se definió como un conjunto de datos \emph{estructurados} dentro del mismo contexto, almacenados sistemáticamente 
para su uso posterior. 
Esta definición es la vista en clase, pero para profundizar primero se establece que un conjunto de datos en bruto 
y dentro del mismo contexto evolucionan o se transforman en información real y útil. Por lo que una base de datos 
es el lygar donde se puede almacenar de manera ordenada y con sentido todos estos datos primigeneos para convertirlos 
así en un pedazo de información útil. 

Se busca que se pueda acceder a los conjuntos de datos de manera sencilla, que sea longevo y contenga solamente la 
información realmente necesaria e importante. Esto nos interesa porque los datos es información, la información nos 
lleva a conocer el contexto y esto nos lleva a la toma de deciciones con inteligencia y bien argumentadas.

\subsection{Menciona una ventaja de las BD}
Las bases de Datos a diferencia de las otras formas de almacenar datos tienen la gran ventaja de facil acceso a la 
información buscada. Esto sucede siempre y cuando se le de un mantenimiento correcto además de una planificación buena 
desde el inicio. 

\subsection{Menciona  una desventaja de las BD}
Una desventaja que va de la mano con la ventaja ya mencionada, es el mantenimiento o su tamaño creciente constante. 
Es impontante que la base de Datos contenga solo los datos más relevantes y es la razón por la que es imperativo revisar, 
constantemente que información entra y cuál hay, ya que es una posibilidad que esta llegue a crecer tanto que ya no haya 
espacio para la información realmente relevante. 

\end{document}
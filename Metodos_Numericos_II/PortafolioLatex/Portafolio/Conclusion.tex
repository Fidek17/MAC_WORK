\documentclass{article}
\usepackage{graphicx}
\usepackage{amsmath}
\usepackage{mathtools}
\usepackage[utf8]{inputenc}
\usepackage{changepage}
\providecommand{\abs}[1]{\lvert#1\rvert}
\providecommand{\norm}[1]{\lVert#1\rVert}
\usepackage{xcolor, colortbl}
\usepackage{array, multirow, multicol}
\usepackage{float}
\pagestyle{empty}

\begin{document}
\section*{Conclusión Del Portafolio Final}
Durante todo el curso para ser sinceros siempre observe cada uno de estos problemas como un reto,
me los tomaba tan enserio que me llevaba un día entero en la normalidad completar uno solo, esto que
no solo tomaba con extrema seriedad la problemática a comprender para poder resolverla, sino que se convirtió en un
estándar para mi.


Recuerdo que al inicio del curso la maestra mencionó que era tarea del estudiante decidir qué hacía o no con su tiempo y
así fue que decidí hacer cada una de estas tarea con la herramienta LaTeX y no me podría sentir más feliz de
hecho ya que como en cada uno de los métodos o ejercicios a realizar, me otorgó una soltura con esta herramienta que
yo estoy seguro que me será útil por mucho tiempo.


Acerca del curso considero que al dar el 100 de mi persona en cada uno de los ejercicios, realmente pude interiorizar
cada uno de los conceptos o al menos tener una buena y sólida base para regresar a ellos de tal manera que solo sea necesario una
investigación breve a este portafolio para poder resolver y entender algún problema de la índole que se presente.
Me agrado hacer el portafolio aunque en algunos momentos me resultase pesado debido a su complejidad y fuerte exigencia
que considero demanda la carrera, pero me voy con un buen sabor de boca de haberme retado y estar seguro de que podré aplicar todos
estos métodos y aprendizajes además de tener un buen trabajo que revisitar si es necesario.


No considero que haya una conclusión más técnica que agregar ya que todo está en las páginas anteriores.




\end{document}
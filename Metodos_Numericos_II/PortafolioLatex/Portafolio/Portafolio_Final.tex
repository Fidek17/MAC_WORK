\documentclass{article}
\usepackage{graphicx}
\usepackage{amsmath}
\usepackage{mathtools}
\usepackage[utf8]{inputenc}
\usepackage{changepage}
\providecommand{\abs}[1]{\lvert#1\rvert}
\providecommand{\norm}[1]{\lVert#1\rVert}
\usepackage{xcolor, colortbl}
\usepackage{array, multirow, multicol}
\usepackage{float}
\pagestyle{empty}

\begin{document}
\begin{titlepage}
    \centering
    {\includegraphics[width=2.5cm]{logo.png}\par}
    {\texttt{\bfseries \LARGE Universidad Nacional Autónoma de México} \par}
    \vspace{1cm}
    {\itshape \Large \bfseries Facultad de Estudios Superiores Acatlán \par}
    \vspace{3cm}
    {\scshape \Huge Portafolio Electrónico \par}
    \vspace {3cm}
    {\slshape \Large Materia: Métodos Numéricos \par}
    \vspace{2cm}
    {\Large Autor: Díaz Valdez Fidel Gilberto\par}
    {\Large Número de cuenta: 320324280\par}
    \vfill
    {\itshape Noviembre 2023 \par}
\end{titlepage}


\tableofcontents
\section{Ejercicio 1: Error de Redondeo ................ 5 }
\subsection{Propósito}
\subsection{Indicaciones}
\subsection{Ejecución}
\subsection{Conclusión}

\section{Ejercicio 2: Métodos de Bisección y Posición Falsa ................................................... 15}
\subsection{Propósito}
\subsection{Instrucciones}
\subsection{Fórmula}
\subsection{Gráfica}
\subsection{Elegir un intervalo de solución}
\subsection{Encontrar una raíz, con tolerancia 0.00005 en
el error relativo y absoluto}
\subsection{Conclusión}

\section{Ejercicio 3: Métodos Abiertos ................ 21}
\subsection{Propósito}
\subsection{Instrucciones}
\subsection{Fórmula}
\subsection{Gráfica}
\subsection{Intervalo de solución}
\subsection{Implementación del método de Newton-Raphson y de la Secante}
\subsection{Conclusión}
\pagebreak
\section{Ejercicio 4: Bairstow ........................... 28}
\subsection{Propósito}
\subsection{Instrucciones}
\subsection{Planteamiento}
\subsection{Gráfica}
\subsection{Implementación del método}
\subsection{Ejecución del método}
\subsection{Conclusión}

\section{Ejercicio 5: Inversa/Determinante ............ 36}
\subsection{Propósito}
\subsection{Instrucciones}
\subsection{Planteamiento}
\subsection{Ejecución}
\subsection{Conclusión}

\section{Ejercicio 6: Método de Gauss .................. 44}
\subsection{Propósito}
\subsection{Instrucciones}
\subsection{Planteamiento}
\subsection{Construcción}
\subsection{Conclusión}

\pagebreak

\section{Ejercicio 8: Método de Gauss-Jordan Particionado ................................................ 50}
\subsection{Propósito}
\subsection{Instrucciones}
\subsection{Planteamiento}
\subsection{Procedimiento}
\subsection{Conclusión}

\section{Ejercicio 9: Métodos Iterativos ................ 58 }
\subsection{Propósito}
\subsection{Instrucciones}
\subsection{Preparativos}
\subsection{Ejecución}
\subsection{Conclusión}

\section{Conclusión Del Portafolio Final ................ 67}




\end{document}
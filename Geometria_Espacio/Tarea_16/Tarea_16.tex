\documentclass{article}
\usepackage{graphicx}
\usepackage{tikz}
\usepackage{pgfplots}
\usepackage{tcolorbox}
\usepackage{xcolor}
\usepackage{changepage}
\usepackage{wrapfig}
\usepackage{lipsum}
\usepackage{amsmath}
\usepackage{amssymb}
\usepackage{amsfonts}
\begin{document}
\begin{titlepage}
    \centering   
    {\includegraphics[width=2.5cm]{logo.png}\par}
    {\texttt{\bfseries \LARGE Universidad Nacional Autónoma de México} \par}
    \vspace{1cm}
    {\itshape \Large \bfseries Facultad de Estudios Superiores Acatlán \par}
    \vspace{3cm}
    {\scshape \Huge Tarea 16: Distancia de un punto a un plano y distancia entre dos planos\par}
    \vspace {3cm}
    {\slshape \Large Materia: Geometria del Espacio \par}
    \vspace{2cm}
    {\Large Autor: Díaz Valdez Fidel Gilberto\par}
    {\Large Número de cuenta: 320324280\par}
    \vfill
    {\itshape Mayo 2024 \par}
\end{titlepage}
Sea $N_1$ el primer dígito de tu número de cuenta, $N_2$ el segundo dígito, $N_3$ el tercero y así
sucesivamente.
\vspace{10pt}

\textbf{1.} Encuentra la distancia del punto dado al plano dado.
\vspace{10pt}

\textbf{Solución:}
\vspace{10pt}

\textbf{a)} $(-3,0,-3);9x+2y+5z=97$.
\vspace{10pt}
$$d(p,\Pi_1) = \frac{\left|9(-3)+2(0)+5(-3)-97\right|}{\sqrt{9^2+2^2+5^2}}$$
$$d(p,\Pi_1) = \frac{\left|-139\right|}{\sqrt{110}}= \frac{139}{\sqrt{110}}$$

\textbf{b)} $(0,0,0);-2x+8z=0$.
\vspace{10pt}
$$d(p,\Pi_2) = \frac{\left|-2(0)+0(0)+8(0)-0\right|}{\sqrt{(-2)^2+0+8^2}}$$
$$d(p,\Pi_2) = \frac{\left|0\right|}{\sqrt{68}}= \frac{0}{\sqrt{68}}$$

\textbf{c)} $(-2,3,4);-3x+6z=-5$.
\vspace{10pt}
$$d(p,\Pi_2) = \frac{\left|-3(-2)+0(3)+6(4)+5\right|}{\sqrt{(-3)^2+0+6^2}}$$
$$d(p,\Pi_2) = \frac{\left|35\right|}{\sqrt{45}}= \frac{35}{\sqrt{45}}$$

\textbf{2.}Determina la distancia entre los siguiente planos.
\vspace{10pt}

\textbf{Solución:}
\vspace*{10pt}

\textbf{a)} $\Pi_1 :x+y+z=2;\Pi_2 :2x+2y+2z=8$.
\vspace{10pt}

Primero nos tenemos que asegurar de que se traten de planos paralelos entre si para calcular su distancia entre ellos:

Los vectores normales son: $n_1 =(1,1,1)$ y $n_2 = (2,2,2)$

\textbf{2.1.1}
$$n_1\cdot n_2 = (1,1,1) \cdot (2,2,2)$$
$$= 1(2)+1(2)+1(2) = 6$$
$\therefore$ Como $6\neq 0$ entonces los planos no son ortogonales. 
\vspace{10pt}

\textbf{2.1.2}
$$1 = 2k$$
$$1 = 2k$$
$$1 = 2k$$

$\therefore$ Como existe $k=\frac{1}{2}$ que hace que se cumplan las igualdades, podemos afirmar que los planos son paralelos.
\vspace{10pt}

\textbf{2.1.3}
$$1 = 2k$$
$$1 = 2k$$
$$1 = 2k$$
$$2 = 8k$$
$\therefore$ Como no existe $k \in \mathbb{R}$ que sea capaz de cumplir todas las igualdades entonces sabemos que no se trata del mismo plano. 

Ahora que conocemos que si se trata de dos planos paralelos, solo resta encontrar un punto dentro otro plano 
 
\end{document}
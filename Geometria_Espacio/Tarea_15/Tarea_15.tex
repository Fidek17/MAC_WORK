\documentclass{article}
\usepackage{graphicx}
\usepackage{tikz}
\usepackage{pgfplots}
\usepackage{tcolorbox}
\usepackage{xcolor}
\usepackage{changepage}
\usepackage{wrapfig}
\usepackage{lipsum}
\usepackage{amsmath}
\usepackage{amssymb}
\usepackage{amsfonts}
\begin{document}
\begin{titlepage}
    \centering   
    {\includegraphics[width=2.5cm]{logo.png}\par}
    {\texttt{\bfseries \LARGE Universidad Nacional Autónoma de México} \par}
    \vspace{1cm}
    {\itshape \Large \bfseries Facultad de Estudios Superiores Acatlán \par}
    \vspace{3cm}
    {\scshape \Huge Tarea 15: La ecuación de un plano\par}
    \vspace {3cm}
    {\slshape \Large Materia: Geometria del Espacio \par}
    \vspace{2cm}
    {\Large Autor: Díaz Valdez Fidel Gilberto\par}
    {\Large Número de cuenta: 320324280\par}
    \vfill
    {\itshape Mayo 2024 \par}
\end{titlepage}
Sea $N_1$ el primer dígito de tu número de cuenta, $N_2$ el segundo dígito, $N_3$ el tercero y así
sucesivamente.
\vspace{10pt}

\textbf{1.} Encuentra el ángulo agudo que forman los planos:
\vspace{10pt}
\begin{center}
    \begin{minipage}[c]{10cm}
        $2N_1x-4N_3y-N_5z=N_7$ y $N_2x-N_4y+2N_6z-11N_8 =0$.
    \end{minipage}    
\end{center}
\vspace{10pt}

\textbf{Solución:}
\vspace{10pt}

Las ecuaciones que tenemos son: $6x-0y-2z=2$ y $2x-3y+8z-88=0$.

El vector normal de $\Pi_1$ $n_1=(6,0,-2)$ y el de $\Pi_2$ $n_2=(2,-3,8)$. 
\vspace*{10pt}

\hspace*{-1cm}\begin{minipage}[c]{0.5cm}
    \textbf{1.1}
    $$\left| n_1 \cdot n_2\right| = \left|6(2)+(0)(-3)+(-2)(8)\right|$$
    $$\left|-4\right| = 4$$
\end{minipage}\hspace*{6cm}\begin{minipage}[c]{8cm}
    \textbf{1.2}
    \vspace{10pt}
    \begin{itemize}
        \item$\|n_1\| = \sqrt{6^2+0+(-2)^2}$
        $$=\sqrt{40}$$
        \item $\|n_2\| = \sqrt{2^2+(-3)^2+8^2}$
        $$= \sqrt{77}$$
    \end{itemize}
\end{minipage}

Por lo tanto el ángulo entre ambos planos es: 
$$\theta = arcos\left(\frac{\left|n_1\cdot n_2\right|}{\|n_1\|\|n_2\|}\right)$$
$$\theta = arcos\left(\frac{4}{\sqrt{40}\sqrt{77}}\right)$$
\vspace*{10pt}

\textbf{2.} Calcula el ángulo entre los siguientes planos:
\vspace*{10pt}

\begin{center}
    \begin{minipage}[c]{12cm}
        \textbf{a)} El plano que pasa por los puntos $(3, 0, 4)$, $(-12, 3, -4)$ y es 
        perpendicular al plano de ecuación $2x + y - z + 6 = 0$.
        \vspace{10pt}
        
        \textbf{b)} El plano que pasa por el punto $(0,2,2)$ y es paralelo al plano de ecuación
        $2x - 3y - 5z + 6 = 0$.
    \end{minipage}
\end{center}
\vspace{10pt}

\textbf{Solución:}
\vspace{10pt}

Resolviendo \textbf{a)} primero, tenemos que el plano es perpendicular a $2x + y - z + 6 = 0$ por lo que esto nos da uno de los dos 
vectores que necesitamos para construir a $\Pi$, $v =(2,1,-1)$, construiremos el segundo vector como la diferencia de los dos puntos que nos dieron, 
por lo tanto tenemos que $u =(3,0,4) -(-12,3,-4) =(15,-3, 8)$. 

Dado un punto $P =(x,y,z)$ recordar que para crear la forma normal de un plano solo es necesario hacer lo siguiente: 
$$(P-p)\cdot(u \text{ x } v)$$
Donde $P$ es el punto generico y $p$ es el punto base o por el que pasa el plano en cuestión: $P-p =(x,y,z)-(3,0,4) = (x-3,y,z-4)$.

Aplicando la fórmula o la proposcición tenenmos:
$$(P-p)\cdot(u \text{ x } v) = \begin{vmatrix}
    x-3 & y &  z-4 \\
    15 & -3 & 8 \\
    2 & 1 & -1 
\end{vmatrix}$$
$$=(-5)(x-3)+31y+21(z-4) = -5x+31y+21z-69 $$

Para comprobar nuestro resultado vamos a revisar que los vectores normales de los dos planos sean efectivamente perpendiculares: 
$$(2,1,-1)\cdot (-5,31,21) = -10+31-21 = -31+31 = 0$$

Como $(2,1,-1)\cdot (-5,31,21) = 0 $ podemos afirmar que si son perpendiculares y que construimos bien nuestro plano, por lo tanto:
$$\Pi_1 : -5x+31y+21z-69 = 0$$

Continuando con \textbf{b)} tenemos el punto $p = (0,2,2)$ y la ecuación del plano paralelo $2x - 3y - 5z + 6 = 0$.

Analizando el problema podemos decir que para que un plano sea paralelo a otro su vector normal debe tener un ángulo de cero con el vector normal del otro plano, que es esencialmente 
lo mismo que decir que ambos vectores normales deben ser el mismo, con este análisis podemos armar nuestro plano usando el mismo vector normal del plano dado que es $n_1 =(2,-3,5)$ como:

$$\Pi_2 : \{(x,y,z) \in \mathbb{R}^2: (2,-3,-5)\cdot (x-0, y-2, z-2 )=0\}$$
$$\Pi_2: \{(x,y,z) \in \mathbb{R}^2: 2x -3y-5z+16 = 0\}$$

Ahora que tenemos nuestros dos planos $\Pi_1$, $\Pi_2$ y sus respectivos vectores normales $n_1 =(-5,31,21)$, $n_2=(2,-3,-5)$ ya estamos en posicición de calcular su ángulo:
$$\left| n_1 \cdot n_2\right| = \left|-5(2)+(31)(-3)+(-5)(21)\right|$$
$$\left|-208\right| = 208$$
\begin{itemize}
    \item$\|n_1\| = \sqrt{(-5)^2+31^2+(21)^2}$
    $$=\sqrt{1427}$$
    \item $\|n_2\| = \sqrt{2^2+(-3)^2+(-5)^2}$
    $$= \sqrt{38}$$
\end{itemize}

Por lo tanto el ángulo entre ambos planos es: 
$$\theta = arcos\left(\frac{\left|n_1\cdot n_2\right|}{\|n_1\|\|n_2\|}\right)$$
$$\theta = arcos\left(\frac{208}{\sqrt{1427}\sqrt{38}}\right)$$
\vspace*{10pt}

\textbf{3.} Determina si los planos siguientes son paralelos, ortogonales, coincidentes (es decir, el
mismo) o ninguno de los anteriores.
\vspace{10pt}

\textbf{Solución:}
\vspace{10pt}

\textbf{a)} $\Pi_1 :x+y+z=2;\Pi_2 :2x+2y+2z=4$.
\vspace{10pt}

Los vectores normales son: $n_1 =(1,1,1)$ y $n_2=(2,2,2)$. 
\vspace{10pt}

\textbf{1.}
$$n_1 \cdot n_2 = (1,1,1)\cdot (2,2,2)$$
$$ =1(2)+1(2)+1(2)=6 \neq 0$$
Por lo tanto no se son ortogonales.
\vspace*{10pt}

\textbf{2.}
$$1 = 2k$$
$$1 = 2k$$
$$1 = 2k$$
Es evidente que si $k = \frac{1}{2} $ se cumplen las igualdades, por lo tanto podemos afirmar que $\Pi_1$ y $\Pi_2$ son paralelas. 
\vspace{10pt}

\textbf{3.}
$$1 = 2k$$
$$1 = 2k$$
$$1 = 2k$$
$$-2 = -4k$$

De igual manera si $k = \frac{1}{2}$ se cumplen las igualdades, por lo que podemos decir que $\Pi_1$ y $\Pi_2$ no solo son paralelas 
sino el mismo plano también.
\vspace*{10pt}

\textbf{b)} $\Pi_1 :x+2y+3z=1;\Pi_2 :2x+4y+6z=2$.
\vspace{10pt}

Los vectores normales son: $n_1 =(1,2,3)$ y $n_2=(2,4,6)$.
\vspace{10pt}

\textbf{1.}
$$n_1 \cdot n_2 = (1,2,3)\cdot (2,4,6)$$
$$ =1(2)+(2)(4)+(3)(6)=28 \neq 0$$
Por lo tanto no se son ortogonales.
\vspace*{10pt}

\textbf{2.}
$$1 = 2k$$
$$2 = 4k$$
$$3 = 6k$$
Es evidente que si $k = \frac{1}{2} $ se cumplen las igualdades, por lo tanto podemos afirmar que $\Pi_1$ y $\Pi_2$ son paralelas. 
\vspace{10pt}

\textbf{3.}
$$1 = 2k$$
$$1 = 2k$$
$$1 = 2k$$
$$-1 = -2k$$

De igual manera si $k = \frac{1}{2}$ se cumplen las igualdades, por lo que podemos decir que $\Pi_1$ y $\Pi_2$ no solo son paralelas 
sino el mismo plano también.
\vspace*{10pt}

\textbf{c)} $\Pi_1 :9x+9y-z=143;\Pi_2 :x-y-10z=-56$.
\vspace{10pt}

Los vectores normales son: $n_1 =(9,9,-1)$ y $n_2=(1,-1,-10)$. 
\vspace{10pt}

\textbf{1.}
$$n_1 \cdot n_2 = (9,9,-1)\cdot (1,-1,10)$$
$$ =9(1)+9(-1)+(-1)(-10)=10 \neq 0$$
Por lo tanto no se son ortogonales.
\vspace*{10pt}

\textbf{2.}
$$9 = 1k$$
$$9 = -1k$$
$$-1 = -10k$$
Es evidente que no hay $k$ que cumpla las tres igualdades, por lo tanto podemos afirmar que $\Pi_1$ y $\Pi_2$ no son paralelas. 
\vspace{10pt}

\textbf{3.}
$$9 = 1k$$
$$9 = -1k$$
$$-1 = -10k$$
$$-143= 56k$$

De igual manera no hay $k \in \mathbb{R}$ que cumpla todas las igualdades, por lo que podemos decir que $\Pi_1$ y $\Pi_2$ no son paralelas 
ni el mismo plano tampoco. 




\end{document}
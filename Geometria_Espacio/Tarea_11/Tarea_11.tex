\documentclass{article}
\usepackage{graphicx}
\usepackage{tikz}
\usepackage{pgfplots}
\usepackage{tcolorbox}
\usepackage{xcolor}
\usepackage{changepage}
\usepackage{wrapfig}
\usepackage{lipsum}
\usepackage{amsmath}
\usepackage{amssymb}
\usepackage{amsfonts}

\begin{document}
\begin{titlepage}
    \centering   
    {\includegraphics[width=2.5cm]{logo.png}\par}
    {\texttt{\bfseries \LARGE Universidad Nacional Autónoma de México} \par}
    \vspace{1cm}
    {\itshape \Large \bfseries Facultad de Estudios Superiores Acatlán \par}
    \vspace{3cm}
    {\scshape \Huge Tarea 11: Distancia y líneas rectas\par}
    \vspace {3cm}
    {\slshape \Large Materia: Geometria del Espacio \par}
    \vspace{2cm}
    {\Large Autor: Díaz Valdez Fidel Gilberto\par}
    {\Large Número de cuenta: 320324280\par}
    \vfill
    {\itshape Mayo 2024 \par}
\end{titlepage}
Sea $N_1$ el primer dígito de tu número de cuenta, $N_2$ el segundo dígito, $N_3$ el tercero y así
sucesivamente.
\vspace{10pt}

\textbf{1.} Calcula la distancia del punto $s = (N_7, -4N_9, 2N_8)$ a la recta que pasa por los puntos $q = (3N_1, N_2, -2N_3)$ y 
$r= (2N_4, -3N_5,-2N_6)$.
\vspace{10pt}

\textbf{Solución:}
\vspace{10pt}

La fórmula para calcular la distancia de un punto dado a una recta es la siguiente:
$$d (P, l) = \frac{\|u \text{ x } (Q-P)\|}{\|u\|}$$
Donde $u$ es el vector de dirección de la recta, $P$ es el punto y $Q$ es el punto por el que pasa la recta. 

Tenemos los datos de $s = (2, 0, 16)$,  $q = (9, 2, 0)$ y $r= (6, -6,-8)$.

La recta en cuestión es: $L_1:\{(9, 2, 0)+t(3, 8, 8): t \in \mathbb{R}\}$.

Por lo tanto aplicando la fórmula tenemos:
$$d(s,L_1) = \frac{\|(3,8,8) \text{ x }(7,2,-16)\|}{\|(3, 8, 8)\|}$$

\begin{minipage}[c]{0.5cm}
    \textbf{1.}
    $$(3,8,8) \text{ x }(7,2,-16) = \begin{vmatrix}
        i & j & k\\
        3 & 8 & 8 \\
        7 & 2 & -16
    \end{vmatrix}$$
    $$= \begin{vmatrix}
        8 & 8 \\
        2 & -16
    \end{vmatrix}i - \begin{vmatrix}
        3 & 8 \\
        7 & -16
    \end{vmatrix}j+ \begin{vmatrix}
        3 & 8 \\
        7 & 2
    \end{vmatrix}k$$
    $$=(-144, 104, -50)$$
\end{minipage}\hspace*{7cm}\begin{minipage}[c]{0.5cm}
    \textbf{2.}
    $$\|(3,8,8) \text{ x }(7,2,-16)\|= \|(-144,104,-50)\|$$
    $$=\sqrt{(-144)^2+104^2+(-50)^2} = \sqrt{34052}$$
\end{minipage}
\vspace{10pt}

\begin{minipage}[c]{0.5cm}
    \textbf{3.}
    $$\|u\| =\|(3,8, 8)\| = \sqrt{3^2+8^2+8^2}$$
    $$= \sqrt{137}$$
\end{minipage}

$\therefore$ La distancia del punto $s$ a la recta $L_1$ es: $d(s,L_1)= \sqrt{\frac{34052}{137}}$.
\vspace{10pt}

\textbf{2.} Calcula la distancia del punto $s = (2N_3, -4N_6, 2N_9)$ a la recta que pasa por el origen y va en dirección de 
$v = (2N_4, -3N_5, -2N_6)$.
\vspace{10pt}

\textbf{Solución:}
\vspace{10pt}

Nuestros datos son: $s = (0, -16, 0)$ y $v = (6, -6, -8)$.
 $$u = P-S =(0,0,0)- (0,-16,0) = (0,16,0)$$

\begin{minipage}[c]{0.5cm}
    \textbf{1.}
    $$v \text{ x } u = \begin{vmatrix}
        i & j & k\\\
        6 & -6 & -8 \\
        0 & 16 & 0
    \end{vmatrix}$$
    $$=(128, 0, 96)$$
\end{minipage}\hspace*{7cm}\begin{minipage}[c]{0.5cm}
    \textbf{2.}
    $$\|v \text{ x } u\| = \|(128, 0, 96)\|$$
    $$= \sqrt{128^2, +0 + 96^2} = \sqrt{25600}$$
\end{minipage}
\vspace{10pt}

\begin{minipage}[c]{0.5cm}
    \textbf{3.}
    $$\|v\| = \sqrt{6^2+ (-6)^2+ (-8)^2}$$
    $$=\sqrt{136}$$
\end{minipage}

$\therefore$ La distancia de la recta $L_1$ al punto $s$ es: $d(s,L_1) = \sqrt{\frac{25600}{136}}$.
\vspace{10pt}

\textbf{3.} Calcula la distancia entre las líneas rectas $l_1$ y $l_2$, donde:
$$l_1: \{(1,0,0)+ s (-1, 1, 0): s \in \mathbb{R}\}$$
$$l_2 : \{(-1,0,0)+t(2,-6,0): t \in \mathbb{R}\}$$
\vspace{10pt}

\textbf{Solución:}
\vspace{10pt}

\begin{minipage}[c]{0.5cm}
    \textbf{1.}
    $$v = (-1, 1, 0)$$
    $$u = (2,-6,0)$$
    $$v \text{ x } u = \begin{vmatrix}
        i & j & k \\
        -1 & 1 & 0\\
        2 & -6 & 0
    \end{vmatrix}$$
    $$=(0, 0, 4)$$ 
    $$\|(0,0,4)\| = \sqrt{0+0+4^2}$$
    $$= \sqrt{16} = 4$$
\end{minipage}\hspace*{7cm}\begin{minipage}[c]{0.5cm}
    \textbf{2.}
    $$w = p_1-p_2 = (1,0,0)-(-1,0,0) =(2,0,0)$$
    $$w \cdot (v \text{ x } u) = \begin{vmatrix}
        2 & 0 & 0 \\
        -1 & 1 & 0\\
        2 & -6 & 0
    \end{vmatrix}$$
    $$=0$$
\end{minipage}
\vspace{10pt}

Aplicando la fórmula para el calculo de distancias entre dos líneas tenemos:
$$d(l_1,l_2) = \frac{\left|(p_1-p_2) \cdot(v \text{ x } u)\right|}{\|v \text{ x } u\|}$$
$$d(l_1,l_2) = \frac{\left|0\right|}{4} = 0$$
$\therefore$ Como $w \cdot (v \text{ x } u) = 0$  podemos afirmar que estas línea se intersectan y es por eso que su distancia es de cero.
\vspace{10pt}

\textbf{4.} Calcula la distancia entre las dos líneas rectas:
$$l_1: \frac{x-3}{27}= \frac{y-1}{21} = \frac{z-8}{33}$$
$$l_2: \frac{x-1}{9} =\frac{y+3}{7} = \frac{z-3}{11}$$
\vspace{10pt}

\textbf{Solución:}
\vspace{10pt}

Tenemos las siguientes rectas: $l_1: \{(3, 1, 8)+t(27,21,33): t \in \mathbb{R}\}$ y $l_2: \{(1,-3,3)+s(9,7,11): s \in \mathbb{R}\}$.

\begin{minipage}[c]{0.5cm}
    \textbf{1.}
    $$v =(27, 21, 33)$$
    $$u =(9,7,11)$$
    $$v \text{ x } u = \begin{vmatrix}
        i & j & k \\
        27 & 21 & 33 \\
        9 & 7 & 11
    \end{vmatrix}$$
    $$=(0, 0, 0)$$
\end{minipage}

Como son rectas paralelas entre si, solo resta revisar que no se trate de la misma:
$$P-Q = (1,-3,3)- (3,1,8) = (-2, -4, -5 )$$
    $$(P-Q) \text{ x } u = \begin{vmatrix}
        i & j & k \\
        -2 & -4 & -5 \\
        9 & 7 & 11
    \end{vmatrix}$$
    $$= \begin{vmatrix}
        -4 & -5 \\
        7 & 11 
    \end{vmatrix}i - \begin{vmatrix}
        -2 & -5 \\
        9 & 11
    \end{vmatrix}j + \begin{vmatrix}
        -2 & -4 \\
        9 & 7
    \end{vmatrix}$$
    $$=(-9, 23, 22)$$

$\therefore$ Como $(P-Q) \text{ x } u\neq 0$ y $ u \text{ x } v= 0 $ podemos afirmar que $L_1$ y $L_2$ son paralelas 
pero no la misma recta.

Como sabemos ahora con seguridad de que no se trata de la misma recta, la distancia entre ellas será la del segmento de recta que 
sea ortogonal a ambas rectas. 

Haciendo un análisis rápido sobre como encontrar la distancia entre dos rectas paralelas podemos llegar al entendido que se desea encontrar 
la distancia del segmento que es tanto ortogonal a $l_1$ como a $l_2$, pero otra manera de plantearlo es encontrar la distancia de cualquier punto 
perteneciente a $l_1$ a la recta $l_2$ pero ya que conocemos que encontrar esta distancia es lo mismo a encontrar el valor del segmento que es ortogonal 
a $l_2$ (ya que este será el más pequeño posible de todos los segmentos del perteneciente a $l_1$ a $l_2$) podemos afirmar con toda seguridad que encontrar 
este segmento ortogonal a $l_1$ y $l_2$ será lo mismo que encontrar la distancia de un punto de cualquiera de las dos rectas hacia la otra.
\par

Aplicando este analisis tenemos entonces: $P = (3, 1, 8)$ como punto que pertenece a $l_1$ y $l_2: \{(1,-3,3)+s(9,7,11): s \in \mathbb{R}\}$ como nuestra línea a calcular 
su distancia con respecto a $P$. 

Tenemos la fórmula para calcular la distancia de un punto a una línea recta dada: 
$$d (P, l) = \frac{\|u \text{ x } (Q-P)\|}{\|u\|}$$
Donde $u$ es el vector de dirección de la recta, $P$ es el punto y $Q$ es el punto por el que pasa la recta. 

Por lo tanto aplicando:
$$Q-P = (1,-3,3)-(3,1,8) =(-2, -4, -5)$$
$$d (P, l_2) = \frac{\|(9,7,11) \text{ x } (-2,-4,-5)\|}{\|(9,7,11)\|}$$

\begin{minipage}[c]{0.5cm}
    \textbf{4.1}
    $$(9,7,11) \text{ x } (-2,-4,-5) =\begin{vmatrix}
        i & j & k \\
        9 & 7 & 11 \\
        -2 & -4 & -5
    \end{vmatrix}$$
    $$=(4, 23, -22)$$
\end{minipage}\hspace*{7cm}\begin{minipage}[c]{0.5cm}
    \textbf{4.2}
    $$\|(9,7,11) \text{ x } (-2,-4,-5)\| = \|(4,23,-22)\|$$
    $$=\sqrt{4^2+23^2+(-22)^2} = \sqrt{1029}$$
\end{minipage}
\vspace*{20pt}

\begin{minipage}[c]{0.5cm}
    \textbf{4.3}
    $$\|(9,7,11)\| = \sqrt{9^2+7^2+11^2}$$
    $$=\sqrt{251}$$
\end{minipage}
\vspace*{20pt}

$\therefore$ La distancia del punto $P$ a la recta $l_2$ y por lo tanto la distancia de $l_1$ a $l_2$ es: $d(P,l_2)=\sqrt{\frac{1029}{251}}=d(l_1,l_2)$.
\vspace{10pt}

\textbf{5.} Sea $l_1$ la recta que pasa por los puntos $q = (3N_1, N_2, -2N_3)$ y $r = (2N_4, -3N_5, -2N_6)$.

Sea la recta $l_2$ que pasa por el punto $p = (N_1, 2N_2, -N_3)$ y va en dirección de $r =(N_7, -4N_9,2N_8)$.

Calcula la distancia entre las rectas $l_1$ y $l_2$.
\vspace{10pt}

\textbf{Solución:}
\vspace{10pt}

Los datos de la primera recta son: $q = (9, 2, 0)$ y $r = (6, -6, -8)$.
$$l_1: \{(9,2,0)+t(3,8,8)\}$$

Los datos de la segunda recta son: $p = (3, 4, 0)$ y $r =(2, 0,16)$.
$$l_2: \{(3,4,0)+ s(2,0,16)\}$$

\begin{minipage}[c]{0.5cm}
    \textbf{1.}
    $$v = (3,8,8)$$
    $$u = (2,0,16)$$
    $$v \text{ x } u = \begin{vmatrix}
        i & j & k \\
        3 & 8 & 8\\
        2 & 0 & 16
    \end{vmatrix}$$
    $$=(128, -32, -16)$$
    $$\|(128, -32, -16)\| = \sqrt{128^2+ (-32)^2+(-16)^2}$$
    $$\sqrt{17664}$$
\end{minipage}\hspace*{7cm}\begin{minipage}[c]{0.5cm}
    \textbf{2.}
    $$w=p_1-p_2 =(9,2,0)-(3,4,0) =(6, -2, 0)$$
    $$w\cdot (v \text{ x } u) = \begin{vmatrix}
        6 & -2 & 0 \\
        3 & 8 & 8\\
        2 & 0 & 16
    \end{vmatrix}$$
    $$= 128(6)+(-2)(-32)+0(-16)= 832$$
\end{minipage}

Usando la fórmula para el calculo de distancias entre rectas entonces tenemos:

Aplicando la fórmula para el calculo de distancias entre dos líneas tenemos:
$$d(l_1,l_2) = \frac{\left|(p_1-p_2) \cdot(v \text{ x } u)\right|}{\|v \text{ x } u\|}$$
$$d(l_1,l_2) = \frac{\left|832\right|}{\sqrt{17664}}$$

\end{document}
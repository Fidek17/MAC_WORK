\documentclass{article}
\usepackage{graphicx}
\usepackage{tikz}
\usepackage{pgfplots}
\usepackage{tcolorbox}
\usepackage{xcolor}
\usepackage{changepage}
\usepackage{wrapfig}
\usepackage{lipsum}
\usepackage{amsmath}
\usepackage{amssymb}
\usepackage{amsfonts}
\begin{document}
\begin{titlepage}
    \centering   
    {\includegraphics[width=2.5cm]{logo.png}\par}
    {\texttt{\bfseries \LARGE Universidad Nacional Autónoma de México} \par}
    \vspace{1cm}
    {\itshape \Large \bfseries Facultad de Estudios Superiores Acatlán \par}
    \vspace{3cm}
    {\scshape \Huge Tarea 12: Representación vectorial de un plano\par}
    \vspace {3cm}
    {\slshape \Large Materia: Geometria del Espacio \par}
    \vspace{2cm}
    {\Large Autor: Díaz Valdez Fidel Gilberto\par}
    {\Large Número de cuenta: 320324280\par}
    \vfill
    {\itshape Mayo 2024 \par}
\end{titlepage}
Sea $N_1$ el primer dígito de tu número de cuenta, $N_2$ el segundo dígito, $N_3$ el tercero y así
sucesivamente.
\vspace{10pt}

\textbf{1.} Encuentra la representación vectorial del plano que pasa por el origen de $\mathbb{R}^3$ y que es generado por los 
vectores $u = (N_1, -N_3, N_5)$ y $v = (N_7, N_8, -N_9)$.
\vspace{10pt}

\textbf{Solución:}
\vspace{10pt}

Los vectores en cuestión son: $u = (3, 0, 2)$ y $v = (2, 8, 0)$ y el punto por el que pasa es $p=(0,0,0)$.

La forma vectorial general: 
$$\{P+sv+tu: s,t \in \mathbb{R}\}$$
$$\{(0,0,0)+s(3,0,2)+t(2,8,0): s,t \in \mathbb{R}\}$$
$$\{(3s+2t,0s+8t,2s+0t): s,t \in \mathbb{R}\}$$
\vspace{10pt}

\textbf{2.} Decide si el punto $p =(-1,1,0)$ está o no en el plano del ejercicio 1.
\vspace{10pt}

\textbf{Solución:}
\vspace{10pt}

Nuestro plano del ejercicio anterior es el siguiente:
$$\{(3s+2t,0s+8t,2s+0t): s,t \in \mathbb{R}\}$$
Para conocer si el punto $p$ se encuentra en el plano se tiene que cumplir el siguiente sistema de ecuaciones:
$$3s+2t = -1.....(1)$$
$$0s+8t = 1.....(2)$$
$$2s +0t = 0......(3)$$

\center\begin{minipage}[c]{5cm}
    \textbf{2.1}
    \vspace{10pt}
    De $(3)$
    $$2s +0t = 0$$
    $$s = 0$$
    Sustituyendo en $(1)$
    $$3(0)+2t = -1$$
    $$2t = -1$$
    $$t = \frac{-1}{2}$$
    Comprobando los valores en $(2)$
    $$0s+8t = 1$$
    $$0(0)+ 8(\frac{-1}{2}) = 1$$
    $$-4 \neq 1$$
\end{minipage}

Como el sistema de ecuaciones no se cumple, podemos afirmar que el punto $p$ no se encuentra en el plano.
\vspace{10pt}

\textbf{3.} Da la representación vectorial del plano de $\mathbb{R}^3$ que pasa por el punto $p = (N_1,N_2,N_3)$ y
que es generado por los vectores $u = (N_4, N_5, N_6)$ y $v = (N_7, N_8, N_9)$.
\vspace{10pt}

\textbf{Solución:}
\vspace*{10pt}

Nuestros datos son: $p = (3,2,0)$, $u = (3, 2, 4)$ y $v = (2, 8, 0)$.
$$\{P + sv + tu : s,t \in \mathbb{R}\}$$
$$\{(3,2,0) + s(2, 8, 0) + t(3, 2, 4) : s,t \in \mathbb{R}\}$$
$$\{(3+2s+3t, 2+8s+2t,0+0s+4t): s,t \in \mathbb{R}\}$$
\vspace{10pt}

\textbf{4.} Decide si el punto $p= (N_4, -2N_5, N_6)$ está o no en el plano del ejercicio 3.
\vspace*{10pt}

\textbf{Solución:}
\vspace{10pt}
$$p =(3, -4, 4)$$
Nuestro plano anterior:
$$\{(3+2s+3t, 2+8s+2t,0+0s+4t): s,t \in \mathbb{R}\}$$
Se debe cumplir el siguiente sistema de ecuaciones:
$$3+2s+3t = 3....(1)$$
$$2+8s+2t = -4....(2)$$
$$0+0s+4t=4....(3)$$

\center\begin{minipage}[c]{5cm}
    \textbf{4.1}
    \vspace{10pt}
    
    De ecuacion $(3)$
    $$0+0s+4t=4$$
    $$t = 1$$
    Sustituyendo en $(2)$
    $$2+8s+2t = -4$$
    $$2+ 8s + 2(1) = -4$$
    $$ 8s +4 = -4$$
    $$8s = -8$$
    $$s = -1$$
    Comprobando en $(1)$
    $$3+2s+3t = 3$$
    $$3+2(-1)+3(1) = 3$$
    $$3 -2 +3 = 3$$
    $$4 \neq 3$$
\end{minipage}

Como no se cumple el sistema de ecuciones podemos afirmar que el punto $p$ no se encuentra en el plano.
\vspace{10pt}

\textbf{5.} Usa la instrucción Superficie o Surface de Geogebra 3D para graficar los planos de los ejercicios 1 y 3. Anexa una imagen de cada plano.
\vspace{10pt}

\begin{figure*}
    \centering
    \includegraphics*[width=15cm]{Ejercicio1.png}
    \caption{Ejercicio 1}
\end{figure*}

\begin{figure*}
    \centering
    \includegraphics*[width=15cm]{Ejercicio3.png}
    \caption{Ejercicio 3}
\end{figure*}




\end{document}
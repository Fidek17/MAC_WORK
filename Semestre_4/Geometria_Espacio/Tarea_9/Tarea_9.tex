\documentclass{article}
\usepackage{graphicx}
\usepackage{tikz}
\usepackage{pgfplots}
\usepackage{tcolorbox}
\usepackage{xcolor}
\usepackage{changepage}
\usepackage{wrapfig}
\usepackage{lipsum}
\usepackage{amsmath}
\usepackage{amssymb}
\usepackage{amsfonts}

\begin{document}
\begin{titlepage}
    \centering   
    {\includegraphics[width=2.5cm]{logo.png}\par}
    {\texttt{\bfseries \LARGE Universidad Nacional Autónoma de México} \par}
    \vspace{1cm}
    {\itshape \Large \bfseries Facultad de Estudios Superiores Acatlán \par}
    \vspace{3cm}
    {\scshape \Huge Tarea 9: Forma vectorial, paramétrica y simétrica de una línea recta \par}
    \vspace {3cm}
    {\slshape \Large Materia: Geometria del Espacio \par}
    \vspace{2cm}
    {\Large Autor: Díaz Valdez Fidel Gilberto\par}
    {\Large Número de cuenta: 320324280\par}
    \vfill
    {\itshape Abril 2024 \par}
\end{titlepage}

Sea $N_1$ el primer dígito de tu número de cuenta, $N_2$ el segundo dígito, $N_3$ el tercero y así
sucesivamente.
\vspace{10pt}

\textbf{1.} Considera $p = (-N_1, -N_2, N_3 )$, $q = (N_4, -N_5, -N_6)$ y $r = (N_7, -N_8, N_9 )$ tres puntos en $\mathbb{R}^3$-
Da la forma vectorial y las ecuaciones paramétricas de la recta que pasa por $p$ y va en la dirección $v = r-q$.
\vspace{10pt}

\textbf{Solución:}
\vspace{10pt}

Los puntos que tenemos son: $p =(-3,-2 , 0)$, $q =(3, -2, -4)$ y $r =(2, -8, 0)$. 
$$v = p-q =(-3,-2, 0) -(3, -2 -4 ) = (-6, 0, 4)$$
Sabemos que la forma vectorial estandar es:
$$\{P + t\vec{v}: t \in \mathbb{R}\}$$
$$\{(-3,-2, 0)+ t(-6, 0, 4): t \in \mathbb{R}\}$$
Por lo tanto la forma paramétrica es:
$$x = -3 + t(-6)$$
$$y =-2 + t(0)$$
$$z = 0 + t(4)$$

\textbf{2.} Escribe la forma vectorial y la forma simétrica de la recta que pasa por los puntos $q = (N_1, N_2,-N_3)$ y $r = (N_4, -N_5, N_6)$.
\vspace{10pt}

\textbf{Solución:}
\vspace{10pt}

Nuestros puntos son: $q = (3, 2,0)$ y $r = (3, -2, 4)$.

Nuestro vector de dirección es:
$$\vec{v} = q - r = (3, 2,0) - (3, -2, 4) = (0, 6, -4)$$

La forma vectorial por lo tanto es, cuando se pasa por P:
$$\{(3,2,0)+t(0,6,-4):t\in \mathbb{R}\}$$

La forma simétrica es:
$$ x = 3; \frac{y-2}{6} = \frac{z-0}{-4}$$

\textbf{3.} Sea $p = (-N_1, -N_2, N_3)$ un punto y $v =(N_4, -N_5,-N_6)$ un vector en $\mathbb{R}^3$. Escribe las ecuaciones 
paramétricas y la forma simétrica de la recta que pasa por $p$ y va en dirección de $v$.
\vspace{10pt}

\textbf{Solución:}
\vspace*{10pt}

Nuestros datos son: $p = (-3, -2, 0)$ un punto y $v =(3, -2,-4)$ un vector en $\mathbb{R}^3$.

\begin{itemize}
    \item Forma paramétrica:
    $$x = -3 +(3)t$$
    $$y = -2 +(-2)t$$
    $$y = 0 +(-4)t$$
    \item Forma simétrica:
    $$\frac{x+3}{3} = \frac{y+2}{-2} = \frac{z -0}{-4}$$ 
    
\end{itemize}
\vspace{10pt}

\textbf{4.} Escribe la forma vectorial y las ecuaciones paramétricas de la recta que pasa por los puntos $q =(2N_1, 2N_2,-N_3)$ y 
$r = (N_4,-4N_5,3N_6 )$.
\vspace*{10pt}

\textbf{Solución: }
\vspace{10pt}

Nuestros datos son: $q =(6, 4,0)$ y $r = (3,-8,12 )$.

El vector de dirección será $q-r$ y nuestro punto será $q$.
$$ q-r =(6,4,0)-(3,-8,12) = (3, 12, -12)$$

\begin{itemize}
    \item Forma vectorial:
    $$\{q+t(q-r): t \in \mathbb{R}\}$$
    $$\{(6,4,0)+t(3,12,-12): t \in \mathbb{R}\}$$
    $$\{(6+3t,4+12t,0-12t): t \in \mathbb{R}\}$$

    \item Ecuaciones paramétricas:
    $$x = 6+3t$$
    $$y = 4+12t$$
    $$z = 0-12t$$
\end{itemize}
\vspace{10pt}

\textbf{5.} Determina si el punto $q = (-N_1, N_2,-N_3)$ está en la línea recta que pasa por le punto $p=(N_4,-N_5,N_6)$ y 
va en la dirección del vector $v = (N_7,N_8,N_9)$.
\vspace{10pt}

\textbf{Solución:}
\vspace*{10pt}

Los datos que tenemos son: $q = (-3, 2,0)$, $p=(3,-2,4)$ y $v = (2,8,0)$

La forma paramétrica nos da las siguientes ecuaciones:
$$x = 3 +2 t$$
$$y = -2 +8t$$
$$z = 4 +0t$$

Ahora intercambiando los valores de $x,y,z$ por las coordenadas del punto $q$ para verificar si se cumplen las igualdades tenemos:
$$-3 = 3 +2 t$$
$$2 = -2 +8t$$
$$0 = 4 +0t$$

Desde la tercera igualdad podemos ver que no se cumple el sistema de ecuaciones, ya que no hay $t \in \mathbb{R}$ que haga que $0 = 4 +0t$, 
que es lo mismo que decir $0 = 4$, por lo tanto podemos afirmar que $q \not\in l_1$. 




\end{document}
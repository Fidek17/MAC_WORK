\documentclass{article}
\usepackage{graphicx}
\usepackage{tikz}
\usepackage{pgfplots}
\usepackage{tcolorbox}
\usepackage{xcolor}
\usepackage{changepage}
\usepackage{wrapfig}
\usepackage{lipsum}
\usepackage{amsmath}
\usepackage{amssymb}
\usepackage{amsfonts}
\begin{document}
\begin{titlepage}
    \centering   
    {\includegraphics[width=2.5cm]{logo.png}\par}
    {\texttt{\bfseries \LARGE Universidad Nacional Autónoma de México} \par}
    \vspace{1cm}
    {\itshape \Large \bfseries Facultad de Estudios Superiores Acatlán \par}
    \vspace{3cm}
    {\scshape \Huge Tarea 13: El plano que pasa por tres puntos y la forma normal de un plano\par}
    \vspace {3cm}
    {\slshape \Large Materia: Geometria del Espacio \par}
    \vspace{2cm}
    {\Large Autor: Díaz Valdez Fidel Gilberto\par}
    {\Large Número de cuenta: 320324280\par}
    \vfill
    {\itshape Mayo 2024 \par}
\end{titlepage}
Sea $N_1$ el primer dígito de tu número de cuenta, $N_2$ el segundo dígito, $N_3$ el tercero y así
sucesivamente.
\vspace{10pt}

\textbf{1.} Encuentra la representación vectorial del plano de $\mathbb{R}^3$ que pasa por los puntos $p = (N_1,N_2,N_3)$, $q = (N_4,-N_5,N_6)$ y $r = (N_7,N_8,-N_9)$.
\vspace{10pt}

\textbf{Solución:}
\vspace{10pt}

Los puntos que tenemos son: $p = (3,2,0)$, $q = (3,-2,4)$ y $r = (2,8,0)$.

Nuestro plano se verá de la siguiente manera: 
$$\{p +s(q-p) +t(r-p): s,t \in \mathbb{R}\}$$

$$u =q -p = (3,-2,4) - (3,2,0) =(0, -4, 4)$$
$$v = r-p=(2,8,0) - (3,2,0) = (-1,6,0)$$

Por lo tanto la forma vectorial de nuestro plano es:
$$\{(3,2,0)+ s\vec{u} +t\vec{v}: s,t \in \mathbb{R}\}$$
$$\{(3,2,0)+ s(0,-4,4) +t(-1,6,0): s,t \in \mathbb{R}\}$$
$$\{(3+0s-t, 2-4s+6t, 0+4s+0t): s,t \in \mathbb{R}\}$$
\vspace*{10pt}

\textbf{2.} Encuentra la representación paramétrica del plano de $\mathbb{R}^3$ que pasa por los puntos $p = (N_1,N_2,N_3)$, $q = (-N_4,N_5,N_6)$ y $r = (N_7,-N_8,N_9)$.
\vspace{10pt}

\textbf{Solución:}
\vspace{10pt}

Los puntos que tenemos son: $p = (3,2,0)$, $q = (-3,2,4)$ y $r = (2,-8,0)$.

El plano se arma de la misma manera que en el ejercicio anterior:
$$u = q-p = (-3,2,4)-(3,2,0)=(-6,0,4)$$
$$v=r-p=(2,-8,0)-(3,2,0)=(-1,-10,0)$$

\begin{itemize}
    \item Forma vectorial:
    $$\{(3,2,0)+s(-6,0,4)+t(1,-10,0):s,t\in \mathbb{R}\}$$
    $$\{(3-6s+t, 2+0s-10t, 0+4s+0t): s,t \in \mathbb{R}\}$$
    \item Forma paramétrica:
    $$x =3-6s+t$$
    $$y = 2+0s-10t$$
    $$z = 0+4s+0t$$
\end{itemize}
\vspace{10pt}

\textbf{3.} Encuentra la representación normal del plano que pasa por el origen de $\mathbb{R}^3$ y que es generado por los vectores 
$u = (N_1, -N_3, N_5)$ y $v = (N_7, N_8, -N_9)$.
\vspace{10pt}

\textbf{Solución:}
\vspace{10pt}

Nuestros vectores son: $u = (3, 0, 2)$ y $v = (2, 8, 0)$.

Vector normal al plano:
$$u \text{ x } v = \begin{vmatrix}
    i & j & k \\
    3 & 0 & 2 \\
    2 & 8 & 0
\end{vmatrix}$$
$$ = (-16, 4, 24)$$

\begin{itemize}
    \item Forma vectorial:
    $$\{(3s+2t,0s+8t,2s+0t): s,t \in \mathbb{R}\}$$
    \item Forma normal:
    
    Si se tiene a $p =(x,y,z)$ entonces la representación normal es:
    $$\{(x,y,z) \in \mathbb{R}^3: (-16,4,24)\cdot((x,y,z)-(0,0,0)) = 0\}$$
    $$\{(x,y,z) \in \mathbb{R}^3: (-16,4,24)\cdot(x,y,z)= 0\}$$
    $$\{(x,y,z) \in \mathbb{R}^3: -16x+4y+24z = 0\}$$
\end{itemize}
\vspace{10pt}

\textbf{4.} Encuentra la representación normal del plano de $\mathbb{R}^3$ que pasa por el punto 
$p = (N_1, -N_2, N_3)$ y que es generado por los vectores $u = (N_4, N_5, N_6)$ y $v = (N_7, N_8, N_9)$.
\vspace*{10pt}

\textbf{Solución:}
\vspace*{10pt}

Los datos que tenemos son: $p = (3, -2, 0)$, $u = (3, 2, 4)$ y $v = (2, 8, 0)$.

Vector normal al plano:
$$u \text{ x } v = \begin{vmatrix}
    i & j & k \\
    3 & 2 & 4\\
    2 & 8 & 0
\end{vmatrix}$$
$$= (-32, 8, 20)$$
Forma normal:

Si se tiene a $p =(x,y,z)$ entonces la representación normal es:
$$\{(x,y,z) \in \mathbb{R}^3: (-32, 8, 20)\cdot((x,y,z)-(3,-2,0)) = 0\}$$
$$\{(x,y,z) \in \mathbb{R}^3: (-32, 8, 20)\cdot((x-3,y+2,z)) = 0\}$$
$$\{(x,y,z) \in \mathbb{R}^3: (-32, 8, 20)\cdot((x-3,y+2,z)) = 0\}$$
$$\{(x,y,z) \in \mathbb{R}^3: -32x+8y+20z+112 = 0\}$$
\vspace{10pt}

\textbf{5.} Encuentra la representación normal del plano de $\mathbb{R}^3$ que pasa por los puntos 
$p = (N_1, N_2, N_3)$, $q = (N_4,N_5,N_6)$ y $r = (N_7,N_8,N_9)$.
\vspace{10pt}

\textbf{Solución:}
\vspace{10pt}

Nuestros puntos son: $p = (3, 2, 0)$, $q = (3,2,4)$ y $r = (2,8,0)$.

$$u = q-p =(3,2,4)-(3,2,0) =(0,0, 4)$$
$$v = r -p =(2,8,0)- (3,2,0) =(-1,6,0)$$

Vector normal al plano:
$$u \text{ x } v = \begin{vmatrix}
    i & j & k \\
    0 & 0 & 4 \\
    -1 & 6 & 0
\end{vmatrix}$$
$$=(-24, -4, 0)$$

Si se tiene a $p_2 =(x,y,z)$ entonces la representación normal es:
$$\{(x,y,z) \in \mathbb{R}^3: (-24,-4,0)\cdot((x,y,z)-(3,2,0)) = 0\}$$
$$\{(x,y,z) \in \mathbb{R}^3: (-24,-4,0)\cdot((x-3,y-2,z)) = 0\}$$
$$\{(x,y,z) \in \mathbb{R}^3: -24x-4y+0z+80 = 0\}$$

\end{document}
\documentclass{article}
\usepackage{graphicx}
\usepackage{tikz}
\usepackage{pgfplots}
\usepackage{tcolorbox}
\usepackage{xcolor}
\usepackage{changepage}
\usepackage{wrapfig}
\usepackage{lipsum}
\usepackage{amsmath}
\usepackage{amssymb}
\usepackage{amsfonts}


\begin{document}
\begin{titlepage}
   \centering   
   {\includegraphics[width=2.5cm]{logo.png}\par}
   {\texttt{\bfseries \LARGE Universidad Nacional Autónoma de México} \par}
   \vspace{1cm}
   {\itshape \Large \bfseries Facultad de Estudios Superiores Acatlán \par}
   \vspace{3cm}
   {\scshape \Huge Tarea 4 \par}
   \vspace {3cm}
   {\slshape \Large Materia: Geometria del Espacio \par}
   \vspace{2cm}
   {\Large Autor: Díaz Valdez Fidel Gilberto\par}
   {\Large Número de cuenta: 320324280\par}
   \vfill
   {\itshape Marzo 2024 \par}
\end{titlepage}


Sea $N_1$ el primer dígito de tu número de cuenta, $N_2$ el segundo dígito, $N_3$ el tercero y así
sucesivamente.
\par


\textbf{1.} Calcula la norma del vector $u = (N_7, -N_8, N_9)$. Si $u$ no es un vector unitario, encuentra
un múltiplo de $u$ que sea unitario.
\par


\textbf{Solución:}
$$\vec{u} = (2, -8, 0)$$
\vspace{10pt}


\begin{minipage}[c]{0.5cm}
   $$\|u\| = \sqrt{2^2+(-8)^2+0}$$
   $$= \sqrt{4+64+0} = \sqrt{68}$$
\end{minipage}\hspace*{4cm}
$\rightarrow$\hspace*{0.5cm}
\begin{minipage}[c]{6cm}
   Gracias a que comprobamos que $\vec{u}$ no es unitario,
   será necesario normalizar.
   \par


   El proceso de normalización es:
   $$\frac{1}{\|\vec{u}\|}\vec{u}$$
\end{minipage}
\vspace{10pt}


$$\frac{1}{\|\vec{u}\|}\vec{u} = \frac{1}{\sqrt{68}}(2,-8,0)= (\frac{2}{\sqrt{68}}, \frac{-8}{\sqrt{68}}, \frac{0}{\sqrt{68}})$$
$$= \|(\frac{2}{\sqrt{68}}, \frac{-8}{\sqrt{68}}, \frac{0}{\sqrt{68}})\| = \sqrt{(\frac{2}{\sqrt{68}})^2+ \sqrt{\frac{-8}{\sqrt{68}}}^2+0}$$
$$= \sqrt{\frac{4}{68}+\frac{64}{64}} = \sqrt{\frac{68}{68}} = \sqrt{1} = 1$$


\textbf{2.} Encuentra la ecuación de la esfera con centro en el origen $(0,0,0)$ $\in$ $\mathbb{R}^3$ y radio $N_9+1$.
\vspace{10pt}


\textbf{Solución:}
Radio $=0+1 = 1$
$$\therefore \mathbb{S}^2={x \in \mathbb{R}^3| x^2+y^2+z^2 = 1}$$


\textbf{3.} Encuentra la ecuación de la esfera con el centro en el origen $(N_7, -N_8, N_9) \in \mathbb{R}^3$ y radio $N_9+1$.
\vspace{10pt}


\textbf{Solución:}
Radio $= 0+1 = 1$
\par
Origen $= (2, -8 ,  0)$
$$ \mathbb{S}^2={x \in \mathbb{R}^3| (x+2)^2+(y-8)^2+(z+0)^2 = 1}$$
$$\therefore (x+2)^2+(y-8)^2+(z+0)^2 = 1$$


\textbf{4.} Demuestra que si $v$ y $w$ son dos vectores de $\mathbb{R}^3$, entonces:
$$\|v-w\|^2 = \|v\|^2+\|w\|^2- 2v\cdot w$$


\textbf{Solución: }
\vspace{10pt}


Sea $\vec{v} = (a,b,c)$ y $\vec{w}=(d,e,f)$.
$$\|v-w\|^2 =\|(a-d,b-e,c-f)\|^2 = \sqrt{(a-d)^2+(b-e)^2+(c-f)^2}^2$$
$$= (a-d)^2+(b-e)^2+(c-f)^2 = a^2-2ad+d^2+b^2-2be+e^2+c^2-2cf+f^2$$


Por otro lado:
$$\|v\|^2+\|w\|^2- 2v\cdot w = \sqrt{a^2+b^2+c^2}^2 +\sqrt{d^2+e^2+f^2}^2-2(a\cdot d+b\cdot e + c\cdot f)$$
$$= a^2+b^2+c^2+d^2+e^2+f^2 -2(ad+be+cf) = a^2+b^2+c^2+d^2+e^2+f^2 -2ad -2be-2cf$$
$$= a^2-2ad+d^2+b^2-2be+e^2+c^2-2cf+f^2$$
$$\therefore \|v-w\|^2 = \|v\|^2+\|w\|^2- 2v\cdot w$$
\vspace{10pt}
$\hfill\blacksquare$
\vspace{10pt}


\textbf{5.} Muestra que se puede determinar el ángulo $\theta$ entre los vectores $v$ y $w$ por medio de
la fórmula:
$$\text{cos }\theta = \frac{\|v+w\|^2-\|v-w\|^2}{4\|v\|\|w\|}$$
\textbf{Ayuda: Desarrolla $\|v+w\|^2-\|v-w\|^2$}
\vspace{10pt}


\textbf{Solución:} \par
Haciendo uso de la ayuda desarrollaremos la parte indicada para llegar a la solución. \par
$$\|v+w\|^2-\|v-w\|^2 = (v+w)\cdot(v+w)-(v-w)\cdot(v-w)$$
$$(v^2+2vw+w^2)-(v^2-2vw+w^2) = 4vw$$
Ahora sustituyendo nuestro nuevo valor de $\|v+w\|^2-\|v-w\|^2$ tenemos:
$$\text{cos }\theta = \frac{4vw}{4\|v\|\|w\|} \rightarrow \text{cos }\theta = \frac{v\cdot w}{\|v\|\|w\|}$$
Y como se sabe esa es la fórmula para encontrar el ángulo de dos vectores, por lo que la proposición es
cierta.
$\hfill\blacksquare$






\end{document}

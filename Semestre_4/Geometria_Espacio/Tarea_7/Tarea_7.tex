\documentclass{article}
\usepackage{graphicx}
\usepackage{tikz}
\usepackage{pgfplots}
\usepackage{tcolorbox}
\usepackage{xcolor}
\usepackage{changepage}
\usepackage{wrapfig}
\usepackage{lipsum}
\usepackage{amsmath}
\usepackage{amssymb}
\usepackage{amsfonts}

\begin{document}
\begin{titlepage}
    \centering   
    {\includegraphics[width=2.5cm]{logo.png}\par}
    {\texttt{\bfseries \LARGE Universidad Nacional Autónoma de México} \par}
    \vspace{1cm}
    {\itshape \Large \bfseries Facultad de Estudios Superiores Acatlán \par}
    \vspace{3cm}
    {\scshape \Huge Tarea 7: El producto vectorial: cálculo de área de triángulos y paralelogramos \par}
    \vspace {3cm}
    {\slshape \Large Materia: Geometria del Espacio \par}
    \vspace{2cm}
    {\Large Autor: Díaz Valdez Fidel Gilberto\par}
    {\Large Número de cuenta: 320324280\par}
    \vfill
    {\itshape Abril 2024 \par}
 \end{titlepage}

 Sea $N_1$ el primer dígito de tu número de cuenta, $N_2$ el segundo dígito, $N_3$ el tercero y así
sucesivamente.
\vspace{10pt}

\textbf{Ejercicio 1:} Calcula la distancia entre la siguiente pareja de vectores: 
\vspace{10pt}

\textbf{a)} $u = (N_1, 2N_2, -N_3)$, $v = (N_4, -N_5, N_6)$
\vspace{10pt}

\textbf{a)} $u =(2N_1, -N_2, -N_3)$, $v = (N_7, -N_8, 2N_9)$
\vspace{10pt}

\textbf{Solución:}
\vspace{10pt}

1. $u = (3, 2(2), 0)$, $v = (3, -2, 4)$
$$d(u,v) = \sqrt{(3-3)^2+(4-(-2))^2+(0-4)^2}$$
$$d(u,v) = \sqrt{0+36+16} = \sqrt{52}$$

2. $u =(2(3), -2, 0)$, $v = (2, -8, 0)$
$$d(u,v) = \sqrt{(6-2)^2+(-2-8)^2+0^2}$$
$$d(u,v) = \sqrt{16+100+0} = \sqrt{116}$$

\textbf{Ejercicio 2:} Calcula $u\text{ x }v$ y $v\text{ x }u$ para cada una de las siguientes parejas de vectores.    
\vspace{10pt}

\textbf{a)} $u = (N_1, 2N_2, -N_3)$, $v = (N_4, -N_5, N_6)$
\vspace{10pt}

\textbf{b)} $u =(2N_1, -N_2, -N_3)$, $v = (N_7, -N_8, 2N_9)$
\vspace{10pt}

\textbf{Solución:}
\vspace{10pt}


\hspace*{-2cm}\begin{minipage}[c]{6cm}
    a) 1.1 $u = (3, 2(2), 0)$, $v = (3, -2, 4)$
    $$u\text{ x }v = \begin{vmatrix}
        i & j & j \\
        3 & 4 & 0 \\
        3 & -2 & 4
    \end{vmatrix}$$
    $$= \left(\begin{vmatrix}
        4& 0 \\
        -2 & 4 
    \end{vmatrix}i -\begin{vmatrix}
        3 & 0 \\
        3 & 4
    \end{vmatrix}j + \begin{vmatrix}
        3 & 4 \\
        3 & -2 
    \end{vmatrix}k \right)$$
    $$= (16i-12j-18k)$$
    $$=(16, -12, -18)$$
\end{minipage}\hspace*{4cm}\begin{minipage}[c]{6cm}
    b) 1.2  $u = (3, 2(2), 0)$, $v = (3, -2, 4)$
    $$v \text{ x }u = \begin{vmatrix}
        i & j & k \\
        3 & -2 & 4 \\
        3 & 4 & 0
    \end{vmatrix}$$
    $$ = \left(\begin{vmatrix}
        -2 & 4 \\
        4 & 0 
    \end{vmatrix}i - \begin{vmatrix}
        3 & 4 \\
        3 & 0
    \end{vmatrix}j +\begin{vmatrix}
        3 & -2 \\
        3 & 4 \\ 
    \end{vmatrix}k\right)$$
    $$=(-16i+12j +18k)$$
    $$=(-16, 12, 18)$$
\end{minipage}
\vspace{10pt}

\hspace*{-2cm}\begin{minipage}[c]{6cm}
    b) 2.1 $u =(6, -2, 0)$, $v = (2, -8, 0)$
    $$u \text{ x } v = \begin{vmatrix}
        i & j & k\\
        6 & -2 & 0\\
        2 & -8 & 0 \\
    \end{vmatrix}$$
    $$= \left(\begin{vmatrix}
        -2 & 0 \\
        -8 & 0 
    \end{vmatrix}i - \begin{vmatrix}
        6 & 0 \\
        2 & 0 
    \end{vmatrix}j+ \begin{vmatrix}
        6 & -2 \\
        2 & -8
    \end{vmatrix}k\right)$$
    $$=(0i -0j-44 k)$$
    $$= (0, 0, -44)$$
\end{minipage}\hspace*{3cm}\begin{minipage}[c]{6cm}
    b) 2.1 $u =(6, -2, 0)$, $v = (2, -8, 0)$
    $$v \text{ x } u = \begin{vmatrix}
        i & j & k \\
        2 & -8 & 0 \\
        6 & -2 & 0 
    \end{vmatrix} $$
    $$= \left(\begin{vmatrix}
        -8 & 0 \\
        -2 & 0\\
    \end{vmatrix}i - \begin{vmatrix}
        2 & 0 \\
        6 & 0 \\
    \end{vmatrix}j +\begin{vmatrix}
        2 & -8 \\
        6 & -2
    \end{vmatrix}\right)$$
    $$=(0i-0j+44k)$$
    $$=(0, 0, 44)$$
\end{minipage}
\vspace{10pt}

\textbf{Ejercicio 3:} Calcula el área del paralelogramo generado por los vectores $u =(N_1, N_2, -N_3)$ y $v = (N_4,-N_5,N_6)$.
\vspace{10pt}

\textbf{Solución:}
$$u =(3, 2, 0)$$
$$v = (3,-2, 4)$$
\vspace{10pt}

\begin{minipage}[c]{0.5cm}
    1.
    $$u \text{ x } v =\begin{vmatrix}
        i & j & k \\
        3 & 2 & 0 \\
        3 & -2 & 4
    \end{vmatrix}$$
    $$= \left(\begin{vmatrix}
        2 & 0 \\
        -2 & 4
    \end{vmatrix}i - \begin{vmatrix}
        3 & 0 \\
        3 & 4 
    \end{vmatrix}j + \begin{vmatrix}
        3 & 2 \\
        3 & -2
    \end{vmatrix}k\right)$$
    $$= (8i -12j -12k)$$
    $$=(8, -12, -12)$$
\end{minipage}\hspace*{7cm}\begin{minipage}[c]{0.5cm}
    2. 
    $$\|u \text{ x } v\| = \|(8,-12,-12)\|$$
    $$ = \sqrt{8^2+(-12)^2+(-12)^2}$$
    $$=\sqrt{64 + 144+144} = \sqrt{352}$$
\end{minipage}
\vspace{10pt}

\textbf{Ejercicio 4:} Calcula el área del triángulo cuyos vértices son los puntos $p = (N_1, N_2, -N_3)$ y 
$q = (N_4, -N_5, N_6)$ y el origen $0 = (0,0,0)$. 
\vspace{10pt}

\textbf{Solución:}
$$u =(3, 2, 0)$$
$$v = (3,-2, 4)$$

Como los vectores son los mismos del ejercicio anterior en el que se calculó el área del paralelogramo que creaban entre si y 
también conocemos que el área del triángulo formado por dos vectores $u$, $v$ no nulos y uno nulo es $\frac{\|u \text{ x }v\|}{2}$.

$\therefore$ Usando este razonamiento y conociendo que $u \text{ x } v = \sqrt{352}$ podemos afirmar que el área del triángulo será igual a 
$\frac{\sqrt{352}}{2}$
$\hfill\blacksquare$
\vspace{10pt}

\textbf{Ejercicio 5:} Calcula el área del paralelogramo cuyos vertices son $p=(11,12,13)$, $q= (10, 10,10 )$, $r= (8,11,12)$ y 
$s = (9, 13,15)$. 
\vspace{10pt}

\textbf{Solución:}
\vspace{10pt}
$$u = p - q = (11,12,13)-(10,10,10) = (1,2,3)$$
$$v = s-q =(9,13,15)- (10,10,10)= (-1,3,5)$$
Ahora ya tenemos dos vectores creados de recorrer al paralelogramo al origen para asi poder calcular su área:
\vspace{10pt}

\begin{minipage}[c]{0.5cm}
    1.
    $$u \text{ x } v =\begin{vmatrix}
        i & j & k \\
        1 & 2 & 3 \\
        -1 & 3 & 5 
    \end{vmatrix}$$
    $$= \left(\begin{vmatrix}
        2 & 3 \\
        3 & 5
    \end{vmatrix}i - \begin{vmatrix}
        1 & 3 \\
        -1 & 5
    \end{vmatrix}j + \begin{vmatrix}
        1 & 2 \\ 
        -1 & 3
    \end{vmatrix}\right)$$
    $$ =(1i-8j+5)$$
\end{minipage}\hspace*{6cm}\begin{minipage}[c]{0.5cm}
    2. 
    $$\|u \text{ x } v\| = \|(1, -8, 5)\|$$
    $$= \sqrt{1^2+(-8)^2+5^2} = \sqrt{1 + 64 + 25}$$
    $$= \sqrt{90}$$
\end{minipage}
\vspace{10pt}

\textbf{Ejercicio 6:} Calcula el área del triángulo cuyos vértices son $p = (11,12,13)$, $q=(10,10,10)$, y $r=(8,11,12)$.
\vspace{10pt}

\textbf{Solución:}
\vspace{10pt}

\begin{minipage}{0.5cm}
    1. 
    $$\vec{p} = p - q= (11,12,13)-(10,10,10) =(1,2,3)$$
    $$\vec{r} = r- q = (8, 11, 12)-(10,10,10) =(-2, 1, 2)$$
\end{minipage}\hspace*{8cm}\begin{minipage}{0.5cm}
    2. 
    $$\vec{p} \text{ x } \vec{r} = \begin{vmatrix}
        i & j & k \\
        1 & 2 & 3 \\
        -2 & 1 & 2
    \end{vmatrix}$$
    $$= \left(\begin{vmatrix}
        2 & 3 \\
        1 & 2
    \end{vmatrix}i - \begin{vmatrix}
        1 & 3 \\
        -2 & 2 
    \end{vmatrix}j + \begin{vmatrix}
        1 & 2 \\
        -2 & 1
    \end{vmatrix}k\right)$$
    $$=(1i -8j+ 5k)$$
    $$=(-1, -7, 5)$$
\end{minipage}
\begin{center}
    \begin{minipage}{0.5cm}
        3. 
        $$\|\vec{p} \text{ x } \vec{r} \| =\|(1,-8,5)\|$$
        $$=\sqrt{(1)^2+(-8)^2+5^2} = \sqrt{1+ 64+ 25}$$
        $$=\sqrt{90}$$
    \end{minipage}
\end{center}
\vspace{10pt}

$\therefore$ El área del triángulo será $\frac{\|\vec{p} \text{ x } \vec{r} \|}{2}$, entonces el área final del triángulo es: 
$$\frac{\|\vec{p} \text{ x } \vec{r} \|}{2} = \frac{\sqrt{90}}{2}$$

\end{document}
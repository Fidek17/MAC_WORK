\documentclass{article}
\usepackage{graphicx}
\usepackage{tikz}
\usepackage{pgfplots}
\usepackage{tcolorbox}
\usepackage{xcolor}
\usepackage{changepage}
\usepackage{wrapfig}
\usepackage{lipsum}
\usepackage{amsmath}
\usepackage{amssymb}
\usepackage{amsfonts}
\begin{document}
\begin{titlepage}
    \centering   
    {\includegraphics[width=2.5cm]{logo.png}\par}
    {\texttt{\bfseries \LARGE Universidad Nacional Autónoma de México} \par}
    \vspace{1cm}
    {\itshape \Large \bfseries Facultad de Estudios Superiores Acatlán \par}
    \vspace{3cm}
    {\scshape \Huge Tarea 14: La ecuación de un plano\par}
    \vspace {3cm}
    {\slshape \Large Materia: Geometria del Espacio \par}
    \vspace{2cm}
    {\Large Autor: Díaz Valdez Fidel Gilberto\par}
    {\Large Número de cuenta: 320324280\par}
    \vfill
    {\itshape Mayo 2024 \par}
\end{titlepage}
Sea $N_1$ el primer dígito de tu número de cuenta, $N_2$ el segundo dígito, $N_3$ el tercero y así
sucesivamente.
\vspace{10pt}

\textbf{1.} Encuentra la representación vectorial del plano que está determinado por la ecuación 
$(N_1 + 1)x = -N_3$.
\vspace{10pt}

\textbf{Solución:}
\vspace{10pt}

La ecuación en cuestion es: $(3+1)x = 0$, es decir $4x+0y+0z+0 = 0$.


Una solución del sistema si $y = 0$  y $z = 1$, entonces $x =0$: $(0,0,1)$


Una solución del sistema si $y = 1$  y $z = 0$, entonces $x =0$: $(0,1,0)$


Otra solución del sistema es si $y=0$, $z = 0$, entonces $x =0$: $(0,0,0)$


Tomaremos a $(0,0,0)$ como el punto base, por lo que los otros vectores son:
$$u = (0,0,1)-(0,0,0) = (0,0,1)$$
$$v = (0,1,0)-(0,0,0) = (0,1,0)$$

Por lo tanto la forma vectorial es:
$$\{(0,0,0)+s(0,0,1)+t(0,1,0): s,t \in \mathbb{R}\}$$
\vspace{10pt}

\textbf{2.} Encuentra la representación vectorial del plano que está determinado por la ecuación 
$(N_1 +1)x-(N_4 +1)z = 0$.
\vspace{10pt}

\textbf{Solución:}
\vspace{10pt}

Nuestra ecuación es: $(3 +1)x-(3 +1)z = 0$, es decir $4x-4z = 0$.


Una solución del sistema si $y = 0$  y $z = 1$, entonces $x =1$: $(1,0,1)$


Una solución del sistema si $y = 1$  y $z = 0$, entonces $x =0$: $(0,1,0)$


Otra solución del sistema es si $y=0$, $z = 0$, entonces $x =0$: $(0,0,0)$


Tomaremos a $(0,0,0)$ como el punto base, por lo que los otros vectores son:
$$u = (1,0,1)-(0,0,0) = (1,0,1)$$
$$v = (0,1,0)-(0,0,0) = (0,1,0)$$

Por lo tanto la forma vectorial es:
$$\{(0,0,0)+s(1,0,1)+t(0,1,0): s,t \in \mathbb{R}\}$$
\vspace{10pt}

\textbf{3.} Encuentra la representación vectorial del plano que está determinado por la ecuación 
$(N_5 + 1)x - (N_7 + 1)y = -N_9$.
\vspace{10pt}

\textbf{Solución:}
\vspace{10pt}

Nuestra ecuación en cuestión es: $(2 + 1)x - (2 + 1)y = 0$, es decir $3x-3y= 0$.
 

Una solución del sistema si $y = 0$  y $z = 1$, entonces $x =0$: $(0,0,1)$


Una solución del sistema si $y = 1$  y $z = 0$, entonces $x =1$: $(1,1,0)$


Otra solución del sistema es si $y=0$, $z = 0$, entonces $x =0$: $(0,0,0)$


Tomaremos a $(0,0,0)$ como el punto base, por lo que los otros vectores son:
$$u = (0,0,1)-(0,0,0) = (0,0,1)$$
$$v = (1,1,0)-(0,0,0) = (1,1,0)$$

Por lo tanto la forma vectorial es:
$$\{(0,0,0)+s(0,0,1)+t(1,1,0): s,t \in \mathbb{R}\}$$
\vspace{10pt}

\textbf{3.} Encuentra la representación vectorial del plano que está determinado por la ecuación 
$-(N_1 + 1)x + (N_2 + 1)y - (N_5 + 1)z = N_7$.
\vspace{10pt}

\textbf{Solución:}
\vspace{10pt}

Nuestra ecuación en cuestión es: $-(3+ 1)x + (2 + 1)y - (2 + 1)y =2$, es decir $-4x+3y-3z= 2$.
 

Una solución del sistema si $y = 0$  y $z = 1$, entonces $x =\frac{-5}{4}$: $(\frac{-5}{4},0,1)$


Una solución del sistema si $y = 1$  y $z = 0$, entonces $x =\frac{1}{4}$: $(\frac{1}{4},1,0)$


Otra solución del sistema es si $y=0$, $z = 0$, entonces $x =\frac{-2}{4}$: $(\frac{-2}{4},0,0)$


Tomaremos a $(\frac{-2}{4},0,0)$ como el punto base, por lo que los otros vectores son:
$$u = (\frac{-5}{4},0,1)-(\frac{-2}{4},0,0) = (\frac{-3}{4}, 0, 1)$$
$$v = (\frac{1}{4},1,0)-(\frac{-2}{4},0,0) = (\frac{3}{4},1,0)$$

Por lo tanto la forma vectorial es:
$$\{(\frac{-2}{4},0,0)+s(\frac{-3}{4},0,1)+t(\frac{3}{4},1,0): s,t \in \mathbb{R}\}$$


\end{document}
\documentclass{article}
\usepackage{graphicx}
\usepackage{tikz}
\usepackage{pgfplots}
\usepackage{tcolorbox}
\usepackage{xcolor}
\usepackage{changepage}
\usepackage{wrapfig}
\usepackage{lipsum}
\usepackage{amsmath}
\usepackage{amssymb}
\usepackage{amsfonts}

\begin{document}
\begin{titlepage}
    \centering    
    {\includegraphics[width=2.5cm]{logo.png}\par}
    {\texttt{\bfseries \LARGE Universidad Nacional Autónoma de México} \par}
    \vspace{1cm}
    {\itshape \Large \bfseries Facultad de Estudios Superiores Acatlán \par}
    \vspace{3cm}
    {\scshape \Huge Tarea 3 \par}
    \vspace {3cm}
    {\slshape \Large Materia: Geometria del Espacio \par}
    \vspace{2cm}
    {\Large Autor: Díaz Valdez Fidel Gilberto\par}
    {\Large Número de cuenta: 320324280\par}
    \vfill
    {\itshape Marzo 2024 \par}
\end{titlepage}

Sea $N_1$ el primer dígito de tu número de cuenta, $N_2$ el segundo dígito, $N_3$ el tercero y así
sucesivamente.

Otra manera de representar un vector $u$ cuyas coordenadas son $(x_1, x_2, x_3)$ es $u = x_1i + x_2j + x_3k$.
\vspace*{10pt}

\textbf{1.} Sea $u=3i-4j-k$,$v=-4i+2j+4k$,$w=i-7j+6k$,$t=-4i+3j-5k.$

$$u = (3, -4, -1 )$$
$$v = (-4, 2, 4)$$
$$w = (1, -7, 6)$$
$$t = (-4, 3, -5 )$$

\begin{minipage}[c]{5cm}
    a) Calcula $t+3w-v$. 
    
    b) Calcula $u\cdot v$.

    c) Calcula $w\cdot(u+v)$.
\end{minipage}\hspace*{3cm}
\begin{minipage}[c]{5cm}
    d) Calcula $(3t-2u)\cdot(5v+2w)$.

    e) Calcula $t\cdot t$.

    f) Calcula $u\cdot w - w \cdot t$.
\end{minipage}
\vspace{10pt}

\begin{minipage}[c]{0.5cm}
    \textbf{a)}
    $$(-4, 3, -5 ) + 3(1, -7, 6)-(-4, 2, 4)=$$
    $$(-4, 3, -5 ) + (3, -21, 18) - (-4, 2, 4)=$$
    $$(-4 +3 +4 , 3 -21 -2 , -5 +18-4)=$$
    $$(3, -20, 9 )$$
\end{minipage}\hspace*{7cm}
\begin{minipage}[c]{0.5cm}
    \textbf{b)}
    $$u\cdot v = (3, -4, -1 )\cdot (-4, 2, 4)$$
    $$= (3)(-4)+(-4)(2)+(-1)(4)$$
    $$=-12-8-4$$
    $$= -24$$
\end{minipage}
\vspace{10pt}

\hspace*{4cm}\begin{minipage}[c]{0.5cm}
    \textbf{d)}
    $$(3t-2u)\cdot(5v+2w) = (3(-4, 3, -5 )-2(3, -4, -1 ))\cdot(5(-4, 2, 4)+2(1, -7, 6))$$
    $$=((-12, 9, -15)+ (-6, 8, 2))\cdot ((-20, 10, 20)+(2, -14, 12))$$
    $$=(-18, 17, -13)\cdot(-18, -4, 32 )$$
    $$=(-18)(-18)+(17)(-4)+(-13)(32)$$
    $$=324-68-416 = -160$$
\end{minipage}
\vspace{20pt}

\begin{minipage}[c]{0.5cm}
    \textbf{c)}
    $$w\cdot(u+v) = (1, -7, 6)\cdot((3, -4, -1 )+ (-4, 2, 4))$$
    $$=(1, -7 , 6) \cdot (-1, -2 , 3)$$
    $$=(1)(-1)+(-7)(-2)+(6)(3)$$
    $$=-1+14+18$$
    $$= 31$$
\end{minipage}\hspace*{7cm}
\begin{minipage}[c]{0.5cm}
    \textbf{e)}
    $$t\cdot t= (-4, 3, -5 )\cdot (-4, 3, -5 )$$
    $$= (-4)^2+ 3^2+ (-5)^2$$
    $$16 +9+ 25$$
    $$= 50$$
\end{minipage}
\newpage
\begin{minipage}[c]{0.5cm}
    \textbf{f)}
    $$u\cdot w - w \cdot t=(3, -4, -1 )\cdot (1, -7, 6)- (1, -7, 6)\cdot (-4, 3, -5 )$$
    $$=(3+28-6)-(-4-21-30)$$
    $$=25+55= 80$$
\end{minipage}
\vspace{10pt}

\textbf{2.} Usa la desigualdad de \emph{Cauchy-Schwarz} para demostrar que cualesquiera números reales
$a$, $b$ y $\theta$ cumplen que: 
$$(a \text{ cos}\theta + b \text{ sin}\theta)^2\leq a^2+b^2$$

\textbf{Ayuda: Usa los vectores v } $= (\text{cos}\theta, \text{sin}\theta,0)$ \textbf{y w} $=(a,b,0)$. 

Conocemos que la desigualdad de \emph{Cauchy-Schwarz} es: $$\left|u\cdot v\right| \leq \|u\|\|v\|$$

Por lo tanto desarrollaremos los vectores que nos proporcionaron para tratar de llegar a un punto donde se pueda 
hacer uso de esa desigualdad.

\begin{minipage}[c]{0.5cm}
    $$\left|u\cdot v\right| = (a \text{ cos }\theta + b \text{ sin }\theta)$$
    $$\|v\| =  \sqrt{\text{cos}^2\theta + \text{sin}^2\theta + 0}$$
    $$\|u\| = \sqrt{a^2+b^2 + 0}$$
\end{minipage}\hspace*{4cm}
$\rightarrow$\hspace*{0.5cm}
\begin{minipage}[c]{0.5cm}
    $$\left|u\cdot v\right|^2 = (a \text{ cos }\theta + b \text{ sin }\theta)^2$$
    $$\|v\|^2 =  \text{cos}^2\theta + \text{sin}^2\theta = 1$$
    $$\|u\|^2 = a^2+b^2$$
\end{minipage}
\vspace{10pt}

Antes de continuar usando la desigualdad se debe hacer notar que $\left|u\cdot v\right| \leq \|u\|\|v\|$ es lo mismo que 
$\left|u\cdot v\right|^2 \leq (\|u\|\|v\|)^2$ ya que se eleva a la segunda potencia en ambas partes de la desigualdad, por lo que 
continua siendo cierta. Una vez dicho esto, como se puede observar en los resultados de haber elevado al cuadrado nuestras igualdades 
anteriores se llega lo que se deseaba demostrar y haciendo uso de la desigualdad de \emph{Cauchy-Schwarz} podemos afirmar que es cierto. 
$$\therefore(a \text{ cos}\theta + b \text{ sin}\theta)^2\leq a^2+b^2$$
$\hfill\blacksquare$
\par 
\textbf{3.} Sean $a$, $b$ y $c$ tres números reales arbitrarios. Muestra que: 
$$\sqrt{a^2+b^2+c^2} \leq \left|a\right|+ \left|b\right|+ \left|c\right|$$
\textbf{Ayuda: Considerar el vector} $(a,b,c) =(a,0,0)+ (0,b,0)+ (0,0,c)$
\vspace{10pt}

\textbf{Demostración.} Sea $\vec{x} = (a, 0, 0)$, $\vec{y} = (0,b,0)$, $\vec{z} = (0, 0, c)$.
Haciendo uso de la siguiente desigualdad: 
$$\|\vec{x}+\vec{y}\| \leq \|\vec{x}\| + \|\vec{y}\|$$
Continuaremos sacando la norma de los tres vectores presentados: 
$$\|\vec{x}+\vec{y}+ \vec{z}\| = \|(a,b,c)\| = \sqrt{a^2+b^2+c^2}$$

Pero también desarrollando para el otro lado: 
\par
\begin{minipage}[c]{0.5cm}
    $$\|\vec{x}\| = \sqrt{a^2+0+0} = \left|a\right|$$
    $$\|\vec{y}\| = \sqrt{b^2} = \left|b\right|$$
    $$\|\vec{z}\| = \sqrt{c^2} = \left|c\right|$$
\end{minipage}\hspace*{4cm}
$\rightarrow$\hspace*{0.5cm}
\begin{minipage}[c]{0.5cm}
    $$\sqrt{a^2+b^2+c^2} \leq \left|a\right|+ \left|b\right|+ \left|c\right|$$
\end{minipage}
\vspace{10pt}

$\hfill\blacksquare$


\textbf{4.} Sea \textbf{$u$} $= (N_1, N_2, -N_3)$ y \textbf{$v$} $= (N_4, -N_5, N_6)$. \textbf{Calcula la norma de $u$, de $v$, de $u+v$ 
y de $5(u-v)$}. 

$$u = (3,2,0)$$
$$v=(3, -2, 4)$$

\textbf{Solución:}
\vspace{10pt}

\begin{minipage}[c]{0.5cm}
    \textbf{a)}
    $$\|u\| = \sqrt{3^2+2^2+0}$$
    $$= \sqrt{9+4+0} =\sqrt{13}$$
\end{minipage}\hspace*{3cm}
\begin{minipage}[c]{0.5cm}
    \textbf{b)}
    $$\|v\| = \sqrt{3^2+ (-2)^2+4^2}$$
    $$= \sqrt{9+4+16} = \sqrt{29}$$
\end{minipage}\hspace*{4cm}
\begin{minipage}[c]{0.5cm}
    \textbf{c)}
    $$\|u+v\| = \|(6, 0, 4)\|$$
    $$= \sqrt{6^2+0^2+4^2} = \sqrt{36+16}$$
    $$=\sqrt{52}$$
\end{minipage}
\vspace{10pt}

\begin{minipage}[c]{0.5cm}
    \textbf{d)}
    $$\|5(u-v)\| = \|5(0, 4, -4)\| = \|(0, 20, -20)\|$$
    $$= \sqrt{20^2+(-20)^2} = \sqrt{800}$$
\end{minipage}


\end{document}
\documentclass{article}
\usepackage{graphicx}
\usepackage{apacite}
\bibliographystyle{apacite}
\usepackage{url}






\begin{document}
\begin{titlepage}
   \centering
   {\includegraphics[width=2.5cm]{logo.png}\par}
   {\texttt{\bfseries \LARGE Universidad Nacional Autónoma de México} \par}
   \vspace{1cm}
   {\itshape \Large \bfseries Facultad de Estudios Superiores Acatlán \par}
   \vspace{3cm}
   {\scshape \Huge Tarea 4: Salario puesto de: administrador de base de datos \par}
   \vspace {3cm}
   {\slshape \Large Materia: Base de Datos \par}
   \vspace{2cm}
   {\Large Autor: Díaz Valdez Fidel Gilberto\par}
   {\Large Número de cuenta: 320324280\par}
   \vfill
   {\itshape Febrero 2024 \par}
\end{titlepage}


\section*{Instrucciones:}
1.  Investiga el salario (promedio) de alguien que realice el trabajo o que cubra un puesto de trabajo como: Administrador de bases de datos


Salario en México


Si en la vacante mencionan tareas o actividades a realizar $\rightarrow$ también agrega esa información


\subsection{Salario Promedio}
Según el portal del Gobierno de México de economía el salario ronda entre los \$11,500.00 mensuales, de igual manera
se trabaja en promedio 35.2 horas con 8,200 de habitantes ocupados en la profesión en el país y un general de 16.7 años de
escolaridad por profesional.


Lo anterior según las cifras recabadas por el Gobierno de México, según un portal dedicado a exponer diferentes vacantes
para trabajos el promedio para el Administrador de Base de Datos en el 2024 es de \$18,000.00 - \$45,000.00 asentándose en un sueldo
de \$24,000.00 por mes. En otro portal donde se publican datos expuestos por trabajadores el promedio está en \$ 14,500.00.


\subsection{Vacante en trabajo relacionado a Base de Datos}


En la plataforma tan conocida \emph{Linkedin} existen un sin número de vacantes pero ninguna viene acompañada con la
información del salario a pagar, pero en este caso daremos por omitido esta pieza de información ya que el promedio ya fue
investigado en diversas páginas y fuentes.


Continuando con la investigación, el primer trabajo mostrado es para un administrador de Base de Datos. Es una empresa llamada
\emph{Macropay} que tiene su ubicación en Yucatán, no divagaré tanto sobre los detalles de la empresa porque considero que se aleja
del objetivo principal del trabajo que es encontrar los requerimientos a cumplir para ser contratado en este ámbito.


Los requisitos son:
\begin{itemize}
   \item Título universitario en Ingeniería en sistemas, informática, carrera afín.
   \item Diseñar, desarrollar y dar mantenimiento (upgrades, parches) a las bases de datos SQL, MySQL, MariaDB, Dynamo DB, SAP Hana (deseable)
   \item Auditorias de Usuarios (roles, perfiles y privilegios)
   \item Creación de bases de datos para nuevas soluciones.
   \item Garantizar el funcionamiento adecuado del sistema (Administración del espacio, Seguridad, Auditoría)
   \item Resolver emergencias de pérdida de información
   \item Realizar un plan de respaldos (DLP, DRP) y mantener siempre disponible la información 24x7.
\end{itemize}


Las responsabilidades:
\begin{itemize}
   \item Diseñar, desarrollar e implementar base de datos de acuerdo con las necesidades de información.
   \item Supervisar el rendimiento de la base de datos, implementar cambios y aplicar nuevos parches y versiones cuando sea necesario.
   \item Garantizar que las bases de datos y los datos tengan una copia de seguridad adecuada y se puedan recuperar correcta y rápidamente en caso de falla.
   \item Determinar, cumplir y documentar las políticas, procedimientos y estándares de las bases de datos.
\end{itemize}


En pocas palabras se le pide al solicitante estudios en el ámbito relacionado con el ámbito de Base de Datos,
disponibilidad para la revisión y mantenimiento de la misma base de datos así como analizar las áreas de oportunidades para
su mejoría, ser capaz de asegurar que la base de datos tenga una copia de seguridad preparada para cualquier eventualidad y
por último, garantizar que la información se encontrará disponible todo el tiempo todos los días de la semana.


\subsection{Conclusión}
En los puntos anteriores se explicaron en generalidad tanto los rangos de salario en México para un administrador
de base de datos como también las tareas que debe ser capaz de realizar para mantener y ganar el puesto además de las condiciones a
cumplir para fungir como candidato. Por lo que se ve en internet, es evidente que hay un cierto auge en este sector y una interesante ventana de
oportunidad laboral a tener en cuenta e investigar si se tiene cierta noción en los conocimientos a necesitar.


\bibliography{ref.bib}


\end{document}
